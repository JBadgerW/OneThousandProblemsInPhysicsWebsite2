MISCELLANSOUS
1. A rifle weighing 5 lbs. discharges a 4-0Z. bullet with a velocity of 100 ft. per second. What will be the velocity of the rifle in the opposite direction?
2. A ball weighing 150 g. and moving at the rate of 30 cm. per second strikes a ball at rest which weighs 400 g. If the smaller ball rebounds with a velocity of 10 cm. per second, what is the velocity of the larger ball after collision?
3. A ball weighing 40 lbs. and moving with a velocity of 30 ft. per second strikes a ball weighing 200 lbs. which is at rest. The smaller ball rebounds with a velocity of 8 ft. per second. What will be the velocity of the larger ball after collision?
4. A mass of 5 lbs., moving with a velocity of 20 ft. per second, meets fairly a body weighing 1 lb. and moving in the opposite direction with a velocity of 75 ft. per second. After collision the mass rebounds with a velocity of 3 ft. per second. What is the velocity of the body after collision?
5. A bullet weighing 2 oz. is shot into a body weighing 30 lbs. hanging freely suspended. If the velocity of the bullet is 1500 ft. per second, what will be the vertical height to which the body will be raised?
6. A wooden pendulum weighing 20 lbs. is struck by an ounce bullet and swung to a vertical height of 2 in. What^ was the velocity of the bullet at the instant of impact?
7. A wooden pendulum weighing 2 lbs. is struck by a bullet weighing 1 oz. and swung to a vertical height of 3 in. With what velocity did the pendulum begin to swing?
8. A Maxim gun delivers 300 i-oz. bullets per minute with a speed of 1600 ft. a second. What force is necessary to hold the gun still? [Lodge.]
9. Calculate the magnitude of the resultant of two forces, of 18 and 20 units respectively, acting from the same point and making with each other an angle of 120°.
10. If a rectangular mass of cork 20 X 4 X 8 cm. is counterpoised in air by 100 g. of metal, find the weight of the cork. Neglect lifting effect of air on metal.
11. Find the height of the homogeneous atmosphere when the mercury barometer reads 76 cm.
12. A barometer tube contains air above the mercury column. On a certain day the mercury stands at 25 in. when the space above it is 6 in. long, and at 24 in. when the space is made only 5 in. long by letting the tube lower down into its cistern. Find the true atmospheric pressure. [Lodge.]
13. A mass of air occupying 5 c.c. is allowed to enter the space above the mercury column in a barometer, and there expands until it occupies 15 c.c. The column of mercury is now 53.33 cm. high. How high was it before the air was allowed to enter?
14. A uniform door 4 ft. wide weighs 50 lbs. The two hinges are 8 ft. apart, and the load is equally divided between them. What is the horizontal pull on the upper hinge? the vertical pull? the resultant of these two? Show by diagram the magnitude and direction of this resultant.
15. What is the height of a mercury barometer when the atmospheric pressure is i.i megadynes per sq. cm.? the height of a water barometer?
16. A rectangular body, the volume of which is to be determined, is measured with a scale, the divisions of which are o.8 their designated length. Will the computed volume be too large or too small? How much?
17. If the density of the body of Problem i6 is to be determined, will the computed value be too large or too small? How much?
18. A cubical box 12 X 12 x 12 in. is ^ full of mercury and f full of water. Find the pressure on one of its sides.
19. How much hot water at 90° C. will be required to change 25 g. of ice at — 20** to water at 20**?
20. How much steam at 100° C. will melt 1000 g. of ice at -10^?
21. A cubical block of uniform density and 10 cm. on a side weighs 10 lbs. It is fastened to a vertical wall by a horizontal pin at the middle of the upper edge of one face of the cube. What is the force tending to pull the pin from the wall?
22. Weights of 7 and 3 lbs. are placed at the ends of a weightless rod and balanced on a pivot. How must the pivot be shifted when the weights are interchanged if equilibrium is still maintained?
23. A body suspended from the two arms of a false balance has apparent weights of 10 and 10^ lbs. What is the ratio of the arms of the balance?
24. A uniform rod is bent into the form of an isosceles triangle, the equal sides being lo cm. long and the base 12 cm. long. Find the center of gravity of the rod.
25. A man ascends a ladder leaning against a wall. Show by means of a diagram whether the ladder will be most likely to slip when the man is at the top or bottom.
26. A uniform rod 3 ft. long, hinged at one end, has a force of 10 lbs. acting vertically upward on the free end. Two ft. from the hinge is suspended a weight of 9 lbs. If the force holds the rod in a horizontal position, what is the weight of the rod?
27. A steelyard formed from a uniform rod 40 in. long and weighing 10 lbs. has a sliding weight of 2 lbs. If the fulcrum is 6 in. from one end, what is the greatest weight that can be weighed by the steelyard?
28. A body passed over 80 ft. in the first 4 seconds and 350 in the first 10 seconds. What was its initial velocity and what its acceleration?
29. A body passes over 300 ft. in 5 seconds and has a final velocity of 100. What was its initial velocity.^
30. A spring balance is suspended by its ring ; a second spring balance is attached to the first by its hook and in an inverted position. The reading of the first balance is 3 lbs. 8 oz. ; that of the second is 3 lbs. What correction must be applied to the second balance when used in a horizontal position? Is it an additive or subtractive correction?
31. The second spring balance of the last problem r6ads, in case of a certain horizontal pull, 9.25 lbs. What is the true reading?
32. A spring balance is suspended by its ring ; a second spring balance is attached to the first by its hook and in an inverted position. The reading of the first spring balance is 3.5 lbs., that of the second 3 lbs. The zero error of the second balance in a vertical position is 0.25 lb. What is the correction for this balance when used in a Horizontal position?
33. If the zero error of the second spring balance of Problem 32 were — 0.25 lb., what would be the correction to apply in case the balance be used in a horizontal position?
34. A square 1 ft. on a side has weights of 2, 4, 6, and 8 g. placed at the corners. Find the center of gravity of the weights.
35. Find the center of gravity of weights of 4, 18, 12, and 14 lbs. placed at the corners of a square 12 in. on a side.
36. A triangular piece of board is in the shape of an isosceles triangle, with its two equal sides 5 ft. long and its base 8 ft. long. A weight of 10 lbs. is hung at its vertex. The weight of the board is 10 lbs. Find the center of gravity of the whole.
37. A room has a volume of 150 cu. yds. The barometer rises from 28 to 30 in. Find how many cu. yds. of air in the room at the higher pressure have entered during the rise.
38. Two bodies start from the same point, one moving east with a uniform velocity of 4 ft. per second, and the other north with a uniform acceleration. If at the end of 5 seconds they are 25 ft. apart, what is the acceleration of the second body .?
39. A body is shot up a frictionless plane rising 3 in 5 with a velocity of 60 ft. per second. How long before it will slide back to its starting place?
40. A body after sliding down a frictionless inclined plane rising 1 in 5 traverses 50 ft. on a frictionless horizontal plane in 4 seconds. What was the length of the inclined plane?
41. Two weights of lo and 6 lbs. hang by a cord passing over a frictionless pulley. With what acceleration will the smaller weight rise?
42. A i-lb. weight hangs over the edge of a smooth table and drags a 15-lb. mass along the surface. Find the acceleration and the tension in the string that joins the two masses.
43. What weight attached to a 6-lb. mass and hanging over the edge of a smooth table will move the mass 6 ft. along the table in 2 seconds?
44. An engine raises a 3-ton cage up a coalpit shaft at the uniform rate of 33 ft. per second. What is the tension in the rope?
45. If the cage of Problem 44 is raised with a uniform acceleration of 6 ft. per second per second, what is the tension in the rope? [Lodge.]
46. A man weighing 150 lbs. goes up and down in an elevator which rises and falls with an acceleration of 5 ft. per second per second. What is his pressure against the bottom of the elevator?
47. A U-tube standing vertically on a table has its arms filled with water to the height of 8 in. A volume of linseed oil of sp. gr. .94, sufficient to fill 8 in. of the tube, is now poured into one of the arms. At what height above the table will be the junction of the oil and water?
48. What weight of lead of sp. gr. 1 1 must be fastened to 5 lbs. of wood of sp. gr. .5, so that the whole shall weigh 1 o lbs. in water?
49. At what depth in the sea will the pressure be double what it is at a depth of 10 ft., the barometer standing at 30 in.?
50. A gas bubble 1 in. in diameter is in water 60 ft. below the surface. What diameter will it have on reaching the surface? The barometer stands at 30 in.
61. On account of the air in the top of a barometer tube 32 in. long, the mercury column stands at 28.8 in. when the true reading is 29.1. What will be the true reading when this barometer indicates a pressure of 29 in.?
52. What will be the resulting temperature obtained by placing 500 g. of mercury of o° C. in 200 g. of water at 60^ C.?
53. A liter of air is measured at 0° C. and 760 mm. pressure. When the temperature is increased loo*' C. and the pressure 240 mm., the volume is found to be 1038.39 c.c. What is the coefficient of expansion of air?
54. Five hundred g. of platinum are dropped into 100 g. of water at i6®.8 C, raising the temperature thereby to 20° C. What was the temperature of the platinum?
55. A ball of lead at 75° C. was dropped into 3 lbs. of water at 1 o° C. The resulting temperature was found to be 29° C. What was the weight of the ball?
56. How much heat is given out when 100 g. of mercury at no° C. are frozen?
57. A ball is shot into the air with a velocity of 320 ft. per second, its path making an angle of 45® with the horizon. With what velocity must a second ball be shot vertically to rise to the same height?
58. A stone is dropped from a balloon ascending with a velocity of 16 ft. per second and reaches the ground in 15 seconds. How high was the balloon when the stone was dropped?
69. If a hole 1 ft. sq. is cut in the bottom of a ship 25 ft. below the surface of the water, what force must be exerted to keep the water out? Consider that a cu. ft. of sea-water weighs 64 lbs.
60. A stone is thrown vertically upward with a velocity that will cause it to rise 144 ft. Two seconds after, another stone is thrown from the same spot in the same direction and with the same velocity. How long after the second stone was thrown and how far above the starting place will the two collide?
61. A uniform rod 30 in. long lying on a smooth table projects 6 in. over the edge of the table. What part of the weight of the rod must a weight be, which, when placed at the extremity of the projecting end, will just cause the rod to tip from the table?