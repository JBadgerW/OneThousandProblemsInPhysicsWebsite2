SOUND
Velocity of sound at 0° C. = 332 m. per second. Increase = 60 cm. per 1° C.
Velocity of sound at 0° C. = 1090 ft. per second. Increase = 2 ft. per 1° C.
1. What time will be required for sound to travel 1 mile (\emph{a}) when the temperature is 0° C.? (\emph{b}) when the temperature is 40° C.?
2. The Eiffel Tower is 1000 ft. high. A man stands on the ground 500 yds. from the center of the base of the tower. If the man's height is neglected, how long after a sound is made at the top of the tower will he hear it? The temperature is 20° C.
3. A church bell is ringing at a distance of $\frac{1}{4}$ mile from one listener and $\frac{1}{2}$ mile from a second. How much louder does it sound to the first than to the second man?
4. On a day when the thermometer is at 0° C. a man riding in a railway train, which moves at the rate of 1 mile a minute, hears a clock directly ahead of him strike with an interval of 1 second between the strokes. What was the actual time between the strokes?
5. On a day when the thermometer stands at 15° C. a stone is dropped from the top of a vertical cliff. The sound caused by its striking the bottom of the cliff reaches a person on the top just 8 seconds after the stone was dropped. What is the height of the cliff?
6. A bullet fired with a velocity of 1200 ft. per second is heard to strike a target 5 seconds after it left the rifle. What is the distance of the target, the temperature being
7. A man, wishing to know the air-line distance between his house and a town, observed that on a day when the thermometer was — 20° C. it took 1 second more for the sound of a whistle in the town to reach him than it did on a day when the thermometer stood at 20° C. What is the distance?
8. On a day when the thermometer stands at 20° C. two reports are made with an intermission of 2 seconds between them. With what velocity must a man be moving away from the point where the reports originated to cause them to reach him with an interval of 3 seconds between them?
9. On a day when the thermometer stands at 13° C. a person riding on a train moving with a velocity of a mile a minute hears a bell directly ahead which seems to produce a note similar to that produced by a tuning fork making 320 vibrations per second. What is the number of vibrations made by the bell?
10. A tuning fork which is known to make 300 vibrations per second is thrown into vibration and placed near another fork which seems to produce the same note. Two beats per second are noticed. What is the rate of vibration of the second fork?
11. What effect will a rise of 20° C. have upon the wave length produced by a tuning fork making 480 vibrations per second?
12. If a tuning fork vibrates 374 times per second, what is the length of its resonance tube on a day when the thermometer stands at — 10° C.?
13. What will be the length of the waves given off by a C, 262 vibrations, tuning fork on a day when the temperature is 20^ C.?
14. If on a day when the temperature is 10° C. a tube 15 in. long gives the best resonance for a tuning fork, what is the vibration number of the fork?
15. A smoked-glass plate is placed vertically in front of a vibrating fork provided with a style. The plate is free to fall under the influence of gravity, and 120 waves are found to be inserted on the plate by the style in the first foot of its length. Determine the vibration number of the fork.
16. What is the length of an open pipe that at 70® F. gives the note C, 262 vibrations?
17. What is the length of a closed pipe which at 60® F. produces the note G, 392 vibrations.?
18. What is the length of a closed tube which at 0° C. will give the greatest reinforcement to the sound of a tuning fork making 256 vibrations per second?
19. What is the length of an open tube that will produce the maximum resonance at 20° C, when a vibrating tuning fork which makes 384 vibrations per second is held near one end?
20. An open pipe which ought to respond to the note E, 330 vibrations, was found to respond to the note F, 350 vibrations. To remedy this, 1 of its length was cut off and a cap put on one of the ends. To what number of vibrations will it now respond?
21. If the G string of a violin is shortened J, what note will be produced when it is thrown into vibration?
22. If the E string of a violin is 15 in. long, how much must it be shortened to produce the note Ejf?
23. If the thickness of a wire is doubled, the length halved, and the tension trebled, what effect will this have upbn the number of vibrations it will make in a given time?
24. A wire 3 ft. long and .2 mm. in diameter, when stretched with a force of 10 lbs., makes 260 vibrations per second upon being thrown into vibration. How many vibrations per second will be made by a wire of the same material 2 ft. long and .1 mm. in diameter if stretched by a force of 4 lbs.?
25. There are two wires of the same material and the same length, but one of them is twice as thick as the other. How much greater force must be applied to the larger wire to cause it to produce when vibrated the same note as the smaller wire produces?
26. A certain cord when stretched by a force of 25 lbs. is found to make 254 vibrations per second. How much must the force be increased to cause it to make 275 vibrations per second?
27. If a vibrating string is found to produce the note C when stretched by a force of 10 lbs., what must be the force exerted to cause it to produce (\emph{a}) the note E? (\emph{d}) the note G?
