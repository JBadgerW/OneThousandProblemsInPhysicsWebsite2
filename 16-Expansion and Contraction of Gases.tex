EXPANSION AND CONTRACTION OF GASES
1. A sealed tube filled with air at 0° C. and 76 cm. pressure is able to withstand a pressure of 10 lbs. to the sq. in. If this tube is placed in the receiver of an air-pump from which the air is gradually exhausted, what will be the height of a barometer placed in the same receiver at the instant the tube breaks.?
2. A barometer in the receiver of an air-pump stands at 10 in. If the piston of the pump has an area of 2 sq. in., how much force will be needed to lift it? The atmospheric pressure is equal to a pressure of 30 in. of mercury.
3. A cu. dm. of gas weighs 6 g. when the barometer stands at 76 cm. Other conditions remaining the same, under what pressure will a cu. dm. of this gas weigh S g.?
4. Mercury is poured into the open arm of a Mariotte's tube until the air in the closed arm is decreased f . What is now the difference in heights of the mercury columns? The barometer stands at 74 cm.
5. A tube 2 m. long, closed at one end, is filled with air and plunged open end down into a cistern of mercury until the volume of the air is decreased ^, How far is the lower end of the tub^ below the surface of the mercury in the cistern?
6. An open tube 50 cm. long is plunged to a depth of 40 cm. in a cistern of mercury, and then the upper end is closed. If the tube is now lifted so that this end is 20 cm. above the level of the mercury in the cistern, what will be the difference of level between the mercury in the tube and that in the cistern?
7. Owing to the pressure of air in the space above the mercury column in a barometer tube 85 cm. long, it is found to indicate a pressure of 70 cm., when an accurate barometer indicates a pressure of 76 cm. What will be the pressure indicated by this barometer when the accurate barometer stands at 72?
8. A gas at constant pressure expands ^^^ of its volume at 0° C. for every degree it is raised above 0° C. How much will it expand for every degree F. above 32° F.?
9. An open vessel was heated until J the air it contained at 0° C. was driven out. How much was it heated?
10. How much must a liter of air at 10° C. be heated in order to increase its volume f?
11. A gas at 0° C. and 760 mm. pressure measured 250 c.c. What will it measure at — 10° C. and 750 mm. pressure?
12. A volume of gas measured 1 cu. ft. at —4° F. and 30 in. pressure. What will be its volume at 68° F. and 39.4 in. . pressure?
13. A flask holding 600 c.c. of air at 0° C. is heated to 25° C. What volume of expanded air escapes? Temperature of escaped air equals 25° C. Neglect expansion of flask.
14. A liter flask filled with air at — 10° C. is heated to 70° C. What will be the volume of the air that escapes if measured at o° C.?
15. A vessel full of air at 0° C. is heated to 80° C, when 2 c.c. of the air measured at o° C. are found to have escaped. How much air was in the vessel before heating t
16. At what temperature, when under a pressure of 75 cm. of mercury, will 2 liters of gas measured at 46° C. and 76 cm. pressure. measure 1.8 liters?
17. If a quantity of gas measured at 0° C. and 76 cm. pressure is subjected to a pressure of 775 mm., how much must the temperature be increased that the volume may remain the same.^
18. If the temperature of a certain volume of air is increased from — 10** to 30° C, how much must the pressure be increased to keep the volume constant?
19. A certain volume of gas is enclosed in a vessel at 76 cm. pressure and —20° C. It is then heated to 40° C. What is the pressure on the sides of the vessel measured in terms of atmospheres?
20. A volume of air at standard temperature and pressure is compressed to \ its original volume and the temperature then raised to 25° C. What will now be the pressure in atmospheres?
21. A liter flask was filled with air at — 10° C. and 750 mm. pressure. If the barometer rises to 760 mm., to what temperature must the flask be raised to drive out \ the air that was in it when it was filled?
22. To how many atmospheres pressure must a liter of gas measured at 76 cm. pressure and —20° C. be subjected to be condensed to ^ a liter when the temperature is 40° C.?
23. A liter of hydrogen at 760 mm. pressure and 0° C. weighs .0896 g. What will 10 liters weigh at —20° C. and 750 mm. pressure?
24. If a liter of nitrogen at o° C. and 76 cm. pressure weighs 1.25 g., how many liters at 25° C. and 74 cm. pressure will be required to weigh 8 g.?
26. The sp. gr. of air at 20° C. normal pressure is .001 18. What will be the weight of a liter of air at — 20° C.?
26. A liter of chlorine gas at o° C. and 76 cm. pressure weighs 3.17 g. If the pressure is decreased to 74 cm., what must the temperature be that a liter of gas may weigh 2 g.?
27. The air in a flexible rubber bag is found to occupy a volume of 1 cu. ft. at 30 in. pressure and 20° C. If the bag is plunged to a depth of 170 ft. in water, the temperature of which is ID° C, what will be its volume?
28. On the top of a mountain the barometer stands at 70 cm. and the temperature is 10° C, while in the valley the barometer stands at 758 mm. and the thermometer at 20° C. What are the relative densities of the air in the two places?
29. The gas enclosed in a piston tube is compressed to ^ its original volume measured at 76 cm. pressure, and the temperature is raised from 10° to 100° C. What is the pressure on each sq. cm. of the piston?
30. A cylinder contains air at 5° C. and 4 atmospheres pressure. Show that if the air is heated to 565° C, the cylinder must be able to stand a pressure of over 12 atmospheres in order not to break.
31. A volume of gas in a graduated flask over water measures 750 c.c. The water in the flask stands 34 cm. above that outside, and the barometer stands at 745 mm. How much will this gas measure if exposed to standard pressure? The sp. gr. of mercury is 13.6.
32. How great a weight will a balloon containing 1000 liters of air at 40° C. raise on a day when the barometer reading is 760 mm. and the temperature o° C.? The balloon itself weighs 100 g. and a liter of air at 0° C. and 760 mm. pressure weighs 1.293 g.
33. A cylinder 48 in. long and closed at one end is filled with air at 77° C. and 30 in. barometric pressure. If the cylinder is sunk open end down to a depth of 68 ft. in water, the temperature at this depth being 14° F., what will be the volume of the air as compared with that at the surface?
34. A tube 1 m. long, closed at one end and filled with air at o° C. and 76 cm. pressure, is plunged open end down into a cistern of mercury at the same temperature. If the lower end of the tube is 75 cm. below the surface of the mercury in the cistern, what is the length of the air column in the tube?