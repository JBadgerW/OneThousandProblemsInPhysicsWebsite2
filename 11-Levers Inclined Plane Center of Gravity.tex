LEVERS, INCLINED PLANE, CENTER OF GRAVITY
1. What weight can be lifted by a lever of the first class, with a power of 200 lbs., the weight-arm being 6 in, and the power-arm 2.5 it I
2. The arms of a lever of the first class are 8 in. and 24 in. What load on the longer arm will balance 150 lbs. on the shorter arm?
3. A man uses an 8-ft. crowbar for lifting a stone of 800 lbs. weight. He thrusts the bar under the stone until the distance from the end of the bar that rests on the earth to the point of contact of bar and stone is 1 ft. With what force must the man lift on the other end of the bar?
4. If the man had raised the stone of Problem 3 by putting one end of the bar under the stone and a fulcrum 1 ft. from that point, what would have been the value, in lbs. weight, of his push down at the other end of the lever?
5. Two men, C and D, carry a load of 500 lbs. on a pole between them. The men are 10 ft. apart, and the load is 4 ft. from C. Find the amount of each man's lift. *
6. The head of a claw-hammer is of such length that the distance from a nail between the claws to the point of contact between the hammer and timber holding the nail is 4 in. The handle of the hammer is 1.5 ft. long. If a force of 25 lbs. is applied at the end of the handle normal to it, what is the force at the nail t
7. A and B sustain upon their shoulders a weight of 300 lbs., placed on a bar 18 ft. long. The weight is placed 12 ft. from A. What is the weight borne by each man?
8. A weightless rod 10 ft. long is balanced at a point 3 ft. from one end. What weight hung from this end will be supported by 1 2 lbs. hung from the other?
9. A uniform lever 10 ft. long balances about a point 1 ft. from one end when loaded at that end with 50 lbs. Find the weight of the lever.
10. A uniform bar 20 ft. long and weighing 100 lbs. rests in a horizontal position on a support 6 ft. from one end. A load of 100 lbs. is suspended from the end of the long arm of the bar. What load applied at the end of the short arm will produce equilibrium?
11. A weightless bar 1 m. in length is graduated in decimeters, and at the end of the ist and 8th graduations are attached 1 and 9 g., respectively. Where is the center of gravity of the combination?
12. A bar 16 ft. long and weighing 20 lbs. is in a horizontal position, and bears at one end a load of 80 lbs. The center of gravity of the bar is 6 ft. from the end carrying the 80 lbs. load. A load of 40 lbs. is placed on the other end of the bar. Where must a fulcrum be placed that the whole may be in equilibrium?
13. A bar 12 ft. long and weighing 12 lbs. has a load of 12 lbs. on one end and a load of 20 lbs. on the other. The whole is in equilibrium when supported at its middle point. Where is the center of gravity of the bar?
14. Two equal weights of 10 lbs. each are hung one at each end of a bar which weighs 5 lbs. and is 6 ft. long. The bar thus weighted balances about a point 3 in. distant from the center of its length. Find its center of gravity.
15. A straight lever 10 ft. long, when unweighted, balances about a point 4 ft. from one end ; but when loaded with 20 lbs. at this end and 4 lbs. at the other, it balances at a point 3 ft. from the end. Find the weight of the lever.
16. On one arm of a false balance a body weighs 1 1 lbs. ; on the other, 17 lbs. 4 oz. What is the true weight.^
17. In one pan of a false balance a piece of meat weighs 1 lb. 8 oz. ; in the other, 2 lbs. 4 oz. What is the true weight?
18. Two weights keep a horizontal, weightless bar at rest. The pressure on the fulcrum is 10 lbs., the difference of the weights 4 lbs., and the difference of the lever arms 9 in. What are the weights and their lever arms?
19. A uniform straight lever 10 ft. long balances at a point 3 ft, from one end, when 1 3 lbs. are hung from this end, and an unknown weight from the other. The lever itself weighs 8 lbs. Find the unknown weight.
20. A uniform beam 12 ft. long rests upon two posts 3 and 5 ft., respectively, from each end. How much must be sawed off from the 5 -ft. end to make the weight supported by its post twice that supported by the other?
21. A bent lever, ACB, has the arm AC, 3 ft. ; CB, 8 ft. ; F, 5 lbs. ; and the angle ACB, 140°. What weight must be attached at B in order to keep AC horizontal? -Pacts from A vertically downward. [Olmstead.^
22. A circular hole 2 in. in diameter is cut in^ uniform circular disc 6 in. in diameter. The centers being 2 in. apart, find the center of gravity of the disc.^
23. Find the center of gravity of weights of 7, 6, 9, and 2 lbs. arranged at the corners of a square 1 ft. on a side.
24. Weights of i, 3, 5, and 7 lbs. are placed at the corners of a uniform square plate which is 10 in. on a side and weighs 4 lbs. Find the center of gravity of the system.
25. A square 1 ft. on a side has weights of i, 2, 3, and 4 lbs. placed at each of its corners. Where is the center of gravity of the weights?
26. A square 1 ft. on a side has weights of 2, 4, 6, and 8 g. placed at the corners. Find the center of gravity of the square.
27. A circular boarc^ 2 ft. in diameter and of uniform thickness has a circular hole 6 in. in diameter cut in it tangent to its circumference. How far from the center of the board is its center of gravity?
28. One of the 4 triangles into which a square is divided by its diagonals is removed. Find the distance of the center of gravity of the remainder from the intersection of the diagonals.
29. A square is divided into 4 equal squares, and one of these is removed. Find the distance of the center of gravity of the remaining portion from the center of the original square.
30. A body weighing 1000 lbs. rests on an incline 5 ft. long and 3 ft. high. What force is required to draw the body up the incline (\emph{a}) if acting parallel to the incline? (\emph{b}) acting parallel to the base? Neglect friction.
31. The height, base, and length of an inclined plane are 6, 8, and 10 ft. What weight will be held on the incline by a force of 100 lbs. (\emph{a}) parallel to the base? (\emph{b}) parallel to the incline? Neglect friction.
32. A mass of 50 lbs. rests on an incline such that it must move 10 ft. in order to rise 6 ft. What -force parallel to the incline will be needed to draw the mass 5 ft.?
33. The bottom of a wagon body is 4 ft. above the ground. A plank 12 ft. long is placed with one end on the ground, the other on the wagon body. What force must a man exert in order to roll a 300-lb. barrel of sugar up this plank? 34. If the force of Problem 33 is exerted parallel to the earth, how large will it need to be?
35. A sled is at rest on a hill that rises 1 ft. in 3. The weight of the sled is 100 lbs. What force parallel to the slope will be required to hold the sled in place? Neglect friction.