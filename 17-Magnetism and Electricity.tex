MAGNETISM AND ELECTRICITY
1. Two small insulated spheres, one charged with 10 units of positive electricity and the other with 10 units of negative electricity, are placed at a distance of 12 cm. apart. How much is the attractive force between them?
2. If one of the spheres in the previous problem had been charged with 15 units of electricity, what would have been their mutual attraction?
3. How far apart must the spheres in the first problem be placed that their mutual attraction shall be ^ dyne?
4. A small metallic sphere is charged with 10 units of electricity. On bringing another similar sphere to a distance of 20 cm. from it, the second sphere is found to be attracted with a force of 1 dyne. What is the charge upon the second sphere?
5. If the two spheres in the previous problem were made to touch each other and then placed at a distance of 4 cm. apart, what would be the mutual force exerted by them? [The spheres have similar electrical capacities.]
6. Two spheres charged one with + 20 and the other with — 15 units of electricity are placed at a certain distance apart. They are then brought into contact and afterward placed in their original position. What is the ratio of the forces acting between them before and after contact?
7. Two small spheres are charged with + 16 and — 4 units of electricity. With what force will they attract each other when at a distance of 4 cm.?
8. If the two spheres of the previous problem are made to touch and then returned to their former positions, with what force will they act on each other? Will this force be attraction or repulsion?
9. When two small metallic spheres are equally charged and placed 8 cm. apart, they are found to repel each other with a force of 2 dynes. How much is the charge on each sphere?
10. A magnetic pole of 80 units strength is 20 cm. distant from a similar pole of 30 units strength. Find the force between them.
11. A magnetic pole of 6 units strength is placed in a field of 0.25 unit strength. What is the force exerted on the pole ^
12. A wire 10 ft. long has a diameter of 1 mm. What ^1 must be the diameter of a wire 40 ft. long to offer (\emph{a}) the same resistance? (b)\ as much resistance?
13. What length of wire .25 mm. in diameter will have the same resistance as 50 m. of wire .75 mm. in diameter.''
14. If the resistance of a wire 5 m. long and 1 mm. in diameter is 2.5 ohms, what will be the resistance of a wire 1 o m. long and .5 mm. in diameter?
16. If the resistance of a wire 10 m. long and .5 mm. in diameter is 8 ohms, what length of the same kind of wire 1 mm. in diameter will be needed to give a resistance of .5 ohm t
16. A piece of wire 5 m. long and .75 mm. in diameter gives 40 ohms resistance. What must be the diameter of a 10 m. piece of wire of the same material in order that its resistance may be only 20 ohms?
17. What e. m. f. is necessary to maintain a current of 10 amperes through a resistance of 40 ohms?
18. If 4 rods of wire weighing 4 lbs. has a resistance of 5 ohms, what will be the resistance of a wire 5 yds. long which weighs 2 lbs.?
19. If 5 yds. of wire weighing 22 oz. has a resistance of 4.7 ohms, what will be the resistance of a wire 10 rods long which weighs 20 lbs.?
20. If 100 ft. of wire weighing 5 lbs. has a resistance of .07 ohm, what will be the resistance of a mile of wire which weighs 1 00 lbs.?
21. If a mile of wire .15 cm. in diameter has a resistance of 2.5 ohms, what is the length of a wire of the same material .05 cm. in diameter which has the same resistance?
22. What are the relative resistances of two wires, one of which is 40 cm. long and weighs 50 g., and the other 25 cm. long and weighs 20 g.?
23. If an ohm is equal to the resistance of a column of mercury 1 06 cm. long and 1 sq. mm. in area of cross-section, what is the resistance of a column of mercury 3 m. long and 4 sq. mm. in area of cross-section?
24. If the conductivities of mercury and copper are as 1 to 50.7, what is the resistance of a copper wire 1 mile long and .35 in. in diameter?
25. If a piece of wire 10 m. long and .25 mm. in diameter offers a resistance of 3 ohms, what length of wire .5 mm. in diameter will be required to give 1.5 ohms resistance?
26. Two wires of the same length and material are found to have resistances of 6 and 10 ohms respectively. If the diameter of the first is .8 mm., what is the diameter of the second?
27. If a mile of copper wire having a diameter of .13 in. has a resistance of 3.25 ohms, what must be the diameter of a copper wire 440 yds. long which has a resistance of 2 ohms?
28. A meter of hard-drawn pure copper wire which weighs .2125 g. offers a resistance of .69 of an ohm. If the resistance of a piece of wire 20 m. long, weighing 12.75 S-> ^^ 1.65 ohms, what is the conductivity of this wire as compared with pure copper wire?
29. A metal wire weighing .188 lb. per ft. and having a diameter of .27 in. was found to have a resistance of .004574 ohm per ft. What is its conductivity as compared with a similar piece of pure copper wire.?
30. What is the diameter of a piece of pure copper wire, 100 miles of which offers a resistance of 120 ohms.? Specific resistance of copper is 1.6 microhms.
31. If the resistance of a platinum wire 5 m. long and weighing 5 g. is 10 ohms, what is the resistance of platinum per c.c, its sp. gr. being 20.33 ^
32. If the specific resistances of zinc and platinum are to each other as 1 is to 2, and a piece of zinc wire 20 ft. long and 1 mm. in diameter has a certain resistance, how long a piece of platinum wire of diameter .3 mm. will have the same resistance?
33. A piece of zinc wire 1 ft. long has a certain resistance. What is the length of a piece of platinum wire of twice the diameter that has the same resistance?
34. The specific resistances of copper and iron are to each other as 1 to 6. How long a copper wire .2 mm. in diameter will have the same resistance as an iron wire 1000 ft. long and .5 ram. in diameter?
35. If the resistance of 1000 ft. of copper wire 2 mm. in diameter is 1.6 ohms, what will be the resistance of 4000 ft. of iron wire 5 mm. in diameter. The conductivity of iron is } that of copper.
36. If the conductivities of iron and copper are as 1 to 6, and the diameters of the two wires, one of iron and the other of copper, are to each other as 3 to 2, what must be their relative lengths if they have the same resistance?
37. If the resistance of a centimeter cube of a certain metal is .125 ohm, what will be the resistance of a wire of this metal 3 m. long and 1 mm. in diameter?
38. The resistance of a wire 2 m. long and J a mm. in diameter is 1 ohm. What is the resistance of two pieces of this kind of wire, each 10 m. long and 1 mm. in diameter, if joined in multiple arc?
39. Two pieces of wire have resistances of 75 and 120 ohms respectively. Find their joint resistance in multiple arc.
40. Three wires whose resistances are 10, 12, and 14 ohms respectively are joined in multiple arc. What is the joint resistance?
41. What is the joint resistance of five pieces of wire in multiple if each piece alone has a resistance of 2 ohms?
42. If the resistance between two points is 25 ohms, what must be the resistance of an additional wire connecting these two points in order to reduce the resistance to 20 ohms?
43. The resistance offered by a piece of wire is 200 ohms. If this piece of wire is drawn out to 4 times its original length, kept of uniform diameter, cut into four pieces of equal length, and these pieces joined in multiple, what will be the resistance?
44. Find the ratio between the resistances of two wires of the same material, one of which is 25 m. long and .15 mm. in diameter, and the other ism. long and .25 mm. in diameter.
46. If the resistance of iron per c.c. is 9.825 microhms, and of German silver per c.c. 21.17 microhms, what will be the resistance offered by an iron wire 30 cm. long and 1 mm. in diameter, and German silver wire 20 cm. long and 2 mm. in diameter, if joined in multiple arc?
46. What is the resistance of the two wires of the previous problem if joined in series?
47. If the specific resistance of copper is to that of iron as 1 to 6, how much iron wire 2.25 mm. in diameter will be required to give the same resistance as 25 m. of copper wire .5 mm. in diameter?
48. A piece of wire 10 m. long and .25 mm. in diameter offers a resistance of 6 ohms. A second piece, whicH is 10 m. long, .5 mm. in diameter, and of specific resistance 6 times that of the first, is joined in multiple. What is the joint resistance?
49. Two wires have resistances of .15 and .05 ohm per foot respectively. If the one having the higher resistance is lo ft. long, what must be the length of the other to make the resistance they offer when connected in multiple arc ^ what it is when they are connected in series?
60. A battery of lo cells is connected in series with an external resistance of 50 ohms. The e. m. f. of each cell is 1.25 volts and its internal resistance 2 ohms. What is the strength of the current?
51. A battery of 20 cells is connected in series with an external resistance of 40 ohms. If the e. m. f. of each cell is I.I volts, and its internal resistance 1.5 ohms, what is the strength of the current?
62. What current would be furnished in the previous .problem if the cells were arranged, (\emph{a}) 4 abreast and 5 in series ; (\emph{b}) 5 abreast and 4 in series?
63. Six cells, each of e. m. f. 1 volt and internal resistance .75 ohm, are to be connected with an external resistance of 1 ohm. (\emph{a}) Show by diagrams the possible methods of arrangement of the cells, (\emph{b}) Prove by numerical work the best arrangement for a maximum current.
54. A current is to be sent through a wire 10 m. long and .25 mm. in diameter. The wire offers a resistance of o.i ohm per 3 linear ft. If 4 cells are employed, each of e. m. f. 1 volt and internal resistance 2 ohms, what is the best arrangement for maximum current?
55. Find the current strength through an external resistance of 10,000 ohms, when a battery of 50 cells, each of e. m. f. 1.25 volts and internal resistance of 2.5 ohms, is arranged for maximum.
56. How would the current in Problem 54 be affected if the area of the battery plates were doubled 1
57 . Suppose it were possible to double the e. m. f . of each of the cells in Problem 54, how would this affect the current?
58. If a cable offers a resistance of 10 ohms per mile, what will be the strength of a current which can be sent through 1000 miles of this cable by a battery of 100 cells, each having an e. m. f. of 1.5 volts and an internal resistance of 4 ohms?
59. Eight cells, each of e. m. f. 1 volt and internal resistance 1.5 ohms, are to be arranged for maximum current through an external resistance of 6 ohms. What is the arrangement of the cells?
60. A battery which generates a current of 5 amperes has its poles connected by 3 wires of the same material, biit whose diameters are 0.5 mm., 0.6 mm., and 0.8 mm. What part of the current flows through each?
61. A battery having an e. m. f. of 8 volts has an internal resistance of 20 ohms. If the current it produces is sent through 5 wires, each having a resistance of 10 ohms, how many amperes current will go through each wire?
62. How many cells, each having an e. m. f. of 1.2 volts and an internal resistance of 4 ohms, will be required to send a current of .045 ampere through the larger of two wires arranged in multiple arc, one of which has a resistance of 600 ohms and the other of 400 ohms?
63. What would be the current sent through the wires in Problem 62, if the cells were arranged in multiple arc?
64. Three wires, each having a resistance of 15 ohms, were joined abreast and a current of 3 amperes sent through them. How much was the e. m. f. of the current?
65. Two wires, each having a resistance of 10 ohms, are arranged abreast and a current of 10 amperes is sent through them. What is the voltage of the current?
66. The e. m. f. of a certain battery is 10 volts and the strength of the current obtained through an external resistance of 4 ohms is 1.25 amperes. What is the internal resistance of the battery?
67. What is the greatest resistance through which 12 cells arranged in series, each having an e. m. f. of 1.5 volts and an internal resistance of .5 ohm, will send a current of 1.5 amperes?
68. A certain number of cells arranged abreast give a current of .145 ampere. The e. m. f. of each cell is 1.5 volts, the internal resistance 6 ohms, and the external resistance of the circuit 10 ohms. How many cells are there?
69. What is the arrangement for maximum current when a battery of 6 cells, each having an e. m. f. of 1 volt and an internal resistance of 4 ohms, is employed to send a current through an external resistance of 4 ohms?
70. What is the arrangement for strongest possible current when 18 cells, each having an e. m. f. of 1 volt and an internal resistance of 3 ohms, are employed to send a current through a wire which offers a resistance of 24 ohms?
71. If the external resistance of Problem 59 had been 35 ohms, what would have been the best arrangement?
72. If the cells in Problem 59 were to send a current through a resistance of 2 ohms, what would be the best arrangement?
73. How many cells, each of e. m. f. 1.5 volts and internal resistance 2 ohms, will be needed to send a current of at least 1 ampere through an external resistance of 40 ohms?
74. What is the greatest current obtainable from a battery of 10 cells connected with an external resistance of 20 ohms, if each cell has an e. m. f. of 1.5 volts and an internal resistance of 5 ohms?
75. What is the best arrangement of 12 cells, each of e. m. f. 1.25 volts and of internal resistance of 2 ohms, for an external resistance of 1000 ohms?
76. How can 24 cells, each having an e. m. f. of 1.8 volts and an internal resistance of 1.5 ohms, be best arranged to send a current through a resistance of 6 ohms?
77. How can 36 cells, each having an e. m. f. of 1.5 volts and an internal resistance of 3 ohms, be best arranged to send a current through an external resistance of 12 ohms?
78. What will be the current strength in Problems 75 and 76?
79. What arrangement of the cells in Problem 76 would send the strongest current through an external resistance of 25 ohms?
80. What would be the best arrangement for the cells in Problem 75 if the external resistance was f of an ohm? How much would the current be?
81. How would you arrange 80 cells, each having an e. m. f . of 2 volts and an internal resistance of 4 ohms, so as to send the strongest possible current through a resistance of 12 ohms? How much will the current be?
82. What is the best arrangement for a battery of 64 cells if it is to send a current through a resistance of 2 ohms? The e. m. f. of each cell is 2 volts and the internal resistance 4 ohms.
83. A battery of 16 similar cells is found when grouped in series to give a voltage of 20. The internal resistance of each cell is 2.5 ohms. Find the strength of the greatest current that can be sent through a resistance of 10 ohms.
84. It is necessary to send a current of 2 amperes through a resistance of 30 ohms. Will a battery of 80 Bunsen cells. each having an e. m. f. of 1.36 volts and an internal resistance of .4 ohm, be able to do this?
85. The resistance of a battery is 3 ohms, and it produces a current of .5 ampere through an external resistance of 1.5 ohms. What is its e. m. f .?
86. What e. m. f. is needed to maintain a current of 4 amperes through a wire that offers a resistance of 5 ohms?
87. Twelve cells arranged 4 in series and 3 abreast, produce J an ampere current through an external resistance of 5 ohms. If the internal resistance of each cell is 3 ohms, what is the e. m. f.?
88. What is the e. m. f. of a cell which has an internal resistance of 2 ohms, if 16 of these cells, arranged 4 in series and 4 abreast, will produce a current of 3 amperes ^
89. Ten cells arranged abreast, each having a resistance of 8 ohms, produce a current of .5 ampere through an external resistance of 2 ohms. What is the e. m. f. of each cell?
90. Two batteries, one of which has a resistance of 30 ohms and the other of 40 ohms, were successively joined in circuit with a galvanometer of no appreciable resistance and a resistance box. With a resistance of 50 ohms the first battery gave a deflection of 30°, but 300 ohms more were needed to cause the second battery to give the same deflection. What is the ratio of the e. m. f. of these batteries?
91. A battery was joined in simple circuit with a 10,000 ohm astatic galvanometer and a resistance box. When 10,000 ohms were placed in the circuit the needle was deflected 3o^ Another battery connected with the same galvanometer required a resistance of 2500 ohms to cause the same deflection. Compare the e. m. f. of the two batteries.
92. How many lamps, each of resistance 20 ohms, and requiring a current of .8 ampere, can be lighted by a dynamo that has an output of 4000 watts?
93. How many lamps like those of Problem 92 can be lighted by a dynamo capable of doing two-horse power of external work?
94. How many lamps, arranged in multiple, can be lighted by a 440-volt dynamo, whose resistance is 4 ohms, if each lanjp requires a current of 1.2 amperes and has a resistance of 20 ohms^