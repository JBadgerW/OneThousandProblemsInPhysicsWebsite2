HARVARD COLLEGE
ADMISSION EXAMINATIONS IN PHYSICS
1890 — 1 900
PHYSICAL SCIENCE
ALTERNATIVE H
[Omit any three questions. If the candidate answers more than five questions he should designate those which he wishes to have regarded as extra ones, otherwise his mark will be five times the average of the marks on the questions answered. Show the processes by which numerical answers are obtained.]
1. A certain body appears by the indications of a spring balance to weigh 15 g. in air and 10 g. in water.
(a) If the balance is correct, what is the specific gravity of the body 1
(b) If each reading of the balance is only ^j^ as large as it should be, what is the specific gravity of the body?
2. If the weight of a c.c. of air is .0012 g. at atmospheric pressure, how many times the ordinary atmospheric pressure would be required to compress air to the density of water, provided the law of compression should hold which your experiments have taught you?
3. Three forces, all applied to the same point, are in equilibrium with each other. If one of these is a force pulling north and another is a force pulling east, show how to find by means of a diagram the direction and magnitude of the third force. Mark the direction of each of the three forces in your figure by means of an arrowhead.
4. A uniform bar 6 ft. long and weighing 8 lbs. is loaded at one end with an 8-lb. weight.
(a) At what distance from this end must a supporting point be placed in order that the whole may balance with the bar in a horizontal position?
(b) If the bar is hinged at the unloaded end, how much work must be done in order to raise the loaded end 2 ft.?
127
5. Describe, with a drawing, the apparatus which you have used in measuring the expansion of a metal bar.
6. Describe any experiment in sound upon which you have spent not less than one hour.
7. In a photographic camera using a single lens let the plate be so placed that the center only of the picture is distinct. Must the plate be pushed nearer the lens or pulled farther away in order that the edges of the picture may become distinct?
8. Let a coil of wire be placed with its plane vertical and extending east and west, and let a current of electricity be flowing in the coil in such a direction that in the top of the coil its course is from east to west. Draw a diagram showing the lines of magnetic force in a horizontal plane cutting through the middle of the coil. Mark each line with arrowheads at several points to indicate the direction of pointing of the magnetic needle at these points. [No more lines need be drawn than the student has himself observed in his laboratory work.]
Admission, (i) 1890.
PHYSICAL SCIENCE
ALTERNATIVE n
[Omit any three questions. Show the processes by which numerical answers are obtained.]
1. A rectangular-sided block of wood 10 cm. thick floats in water with 6 cm. of its depth submerged. Find specific gravity of block. Show your process of reasoning.
2. In determining the specific gravity of air by weighing a bottle, (\emph{a}) when filled with air, (\emph{b}) when empty, why is it important to have the inside of the bottle dry?
3. A door 8 ft. tall, 3 ft. wide, and weighing 50 lbs. is attached to a wall by one edge in an upright position at two points — the first 1 ft. above the bottom, the other 1 ft. below the top.
(a) How great is the total downward pull iJpon the wall at
PHYSICAL SCIENCE 1 29
each point? (\emph{b}) How great is the horizontal force exerted on the wall at each point, and is this force a push or a pull?
4. A ball weighing 20 g. moving with a velocity of 50 cm. per second strikes centrally a ball weighing 100 g. which is at rest. After the collision the larger ball moves north with a velocity of 12 cm. per second. What velocity has the smaller ball after the collision, and in what direction is it moving?
5. A volume of air at 27° C. and under a pressure of 75 cm. of mercury contains 1000 c.c. What would be its volume at 127° C. under a pressure of 1 50 cm. of mercury?
6. Describe carefully the process you would follow in finding the dew-point, and show why the dew-point is important for weather predictions.
7. Draw a diagram illustrating the use of a lens for throwing upon a screen the image of an object. Mark the principal focus of the lens and show how to find the position and size of the object.
8. With a two-fluid galvanic cell, using zinc and copper plates, is the current supposed to go from zinc to copper, or from copper to zinc, inside the cell? outside the cell? Does the zinc plate grow lighter or heavier while the current flows? Does the copper plate grow lighter or heavier? If a magnetic needle is placed just beneath a straight wire carrying a current from south to north, in what direction will the needle point?
Admission, (i) 1891.
PHYSICAL SCIENCE ALTERNATIVE H : EXPERIMENTAL PHYSICS
[Omit any three questions. If the candidate answers more than five questions, he should indicate which ones he wishes to be considered as extra ones. Indicate the numerical processes by which answers are obtained.]
1. The deflections of different rods of the same material under equal loads are proportional to the cubes of their lengths and inversely proportional to their widths and the cubes of their depths-
If a rod 100 cm. long, 2 cm. broad, and 3 cm. thick is deflected 0.5 cm. when placed horizontal and loaded at the middle with a certain weight, what would be the deflection, under the same load, of a rod 50 cm. long, 2 cm. broad, and 1 cm. thick?
2. A cubical vessel, each side of which is 10 cm. square, has a tube 1 sq. cm. in cross-section and 20 cm. tall rising from the middle of its top. The tube is open at both ends, and both vessel and tube are full of water. Neglecting atmospheric pressure and weight of vessel and tube, find
(a) How great is the total pressure which the vessel as now filled exerts upon its support.
(b) How great is the total pressure exerted against the bottom of the vessel by the water within it.
If these pressures are not equal, explain the difference.
3. A cork of specific gravity .25, the volume of which is 10 c.c, floats upon mercury of specific gravity 13.6. How great is the volume of the submerged part of the cork?
4. A door 10 ft. tall, 5 ft. wide, and weighing 100 lbs. is attached to a wall by one edge in an upright position at two points — the first 1 ft. above the bottom, the other 1 ft. below the top.
(a) How great is the total downward pull upon the wall at the two supporting points?
(b) How great is the horizontal force exerted upon the wall at each point, and is this force a push or a pull?
5. The coefficient of linear expansion of steel being .000012, what is the length at 0° C. of a bar which is just 1 m. long at 20° C.?
6. Describe carefully any process which you would recommend for determining the rate of vibration of a tuning fork.
7. The image of a clock-face is thrown upon a screen. The time is 1 2.30. Make a drawing of the image as seen by an observer looking from the lens.
8. A galvanic cell such as you have used is employed to send a current of electricity through 1 m. of No. 30 German-silver wire, and a perfectly similar cell to send a current through 2 m. of the same
PHYSICAL SCIENCE I3I
kind of wire. Tell why the strengths of the. two currents would not be exactly as 2 to 1 .
If 50 m. of the wire were substituted for the 1 m., and 100 m. for the 2 m., would the strengths of the currents be more nearly as 2 to I? Explain.
Admission, (i) 1892.
PHYSICAL SCIENCE ALTERNATIVE U : EXPERIMENTAL PHYSICS
[Omit any three questions. Indicate the processes by which numerical answers are obtained.]
1. A certain beam 4 ft. long, placed horizontal and supported at the ends, is bent downward 0.5 in. by a load placed at the middle. How far would it be bent by the same load if it were 8 ft. long?
2. A vessel is filled with water to a depth of 40 cm. A cylinder of wood 30 cm. long and 100 sq. cm. in area of cross-section, the specific gravity of which is 0.5, extends upward through a hole in the bottom of the vessel, the top of the cylinder being 20 cm. beneath the surface of the water. Show whether the cylinder tends to rise or to fall and how great a force is required to hold it in its present position.
3. From the following data find the specific gravity of sulphuric acid :
Weight of bottle empty 50 g.
Weight of bottle filled with water 1,50 g.
Weight of bottle filled with sulphuric acid . . . 234 g.
How much would this result have been changed if 1 c.c. of the bottle had been left empty when it was weighed with water?
4. A uniform lever 6 ft. long and weighing 20 lbs. lies horizontally across a fulcrum 2 ft. from one end. A mass of 100 lbs. is suspended from the end of the short arm of the lever. How great must be the force applied at the end of the other arm in order that there may be equilibrium?
5. From the following data find the temperature after mixing :
Weight of water used loo g.
Weight of mercury used looo g.
Original temperature of water io°
Original temperature of mercury ioo°
Specific heat of mercury 0333
Number of heat units absorbed by the calorimeter . 80
6. Describe and explain the phenomena seen in boiling water, showing why the temperature of boiling depends upon the pressure.
7. Describe carefully the process by which you have found the shape and size of a real image formed by a lens.
8. Given 8 galvanic cells, each having an electromotive force of 1 volt and a resistance of 2 ohms, with an external resistance of 20 ohms, what will be the strength of the current when the cells are arranged in series^ that is, with the zinc of one cell joined to the copper of the next? When in multiple arc^ that is, with the zincs joined as one and the coppers joined as one t
Admission, (i) 1893.
PHYSICAL SCIENCE
ALTERNATIVE H: EXPERIMENTAL PHYSICS
[Omit any three questions. In numerical questions indicate the process by which the answers are obtained.]
1. A pump is used to draw water from a well through a vertical pipe. How long may the pipe be, the barometer reading 76 cm. and the specific gravity of mercury being 13.6? Tell and explain what would happen if a small hole were bored in the wall of this pipe, when full of water, at a point halfway up.
2. A body weighs 50 g. alone in air. It is then tied to a piece of lead which weighs 100 g. alone in water, and the two together weigh 1 20 g. in water. What is the specific gravity of the body?
3. Define work.
A body weighing 100 lbs. rests upon an incline such that the body must move 10 ft. along the incline in order to rise 8 ft How
PHYSICAL SCIENCE 1 33
much work is required to draw the body 30 ft. along the incline when there is no friction? How much work is required to draw it the same distance when the coefl&cient of friction is J?
4. From the following data find the coefficient of expansion of a gas :
Vol. at 0° C 200 C.C.
Vol. at 100° C 275 CO.
Pressure (by mercury gauge) 76 cm. throughout the experiment. 6. What temperature on the Centigrade scale corresponds to -20® Fahrenheit?
6. Describe what you consider the best work requiring measurements that you have done in sound.
7. The focal length of a certain lens is 20 cm. An arrow 4 cm. long is placed 60 cm. from this lens, its direction being at right angles with the line drawn from it to the lens. What is the length of the image measured straight from tail to tip?
8. What is the object of amalgamating the zinc of a galvanic battery? State and explain the main advantage which two-fluid galvanic cells possess over single-fluid cells.
Admission, (i) 1894.
PHYSICAL SCIENCE ALTERNATIVE U: EXPERIMENTAL PHYSICS
[Answer any five questions. In numerical questions indicate the process by which the answers are obtained.]
1. A cube of iron, each edge of which is 10 cm. long, floats upright in mercury. The density of the iron is y.s, that of the mercury is 13.5. How high does the top of the cube float above the surface of the mercury?
2. A hole 1 sq. cm. in area of cross-section is opened in the wall of a water reservoir 30 m. below the water surface. A person undertakes to prevent leakage by covering the hole with his thumb. How great is the force required?
3. A uniform bar lo ft. long leans with one end against a vertical wall at an angle of 45° ; the other end rests upon the ground. The bar weighs 20 lbs. There is no friction between the bar and the wall, so that the force there exerted is entirely horizontal. How great is this force?
Illustrate the reasoning by means of a diagram.
4. If the rails of a car track vary 60° C. in temperature in the course of a year, and if their least length is 10 m., how much does the gap between the ends of two rails vary during the year, the coefficient of expansion being .000013 ^
6. If 1 g. of coal in burning gives out 7000 units of heat, and if the latent heat of melting for water is 80, how many grams of ice, taken at 0° may be melted and raised to 20° C. by burning 1 g. of coal?
6. Describe " beats " of musical sounds. How and with what apparatus may they be produced?
7. Two plane mirrors are placed parallel, facing each other, and 2 ft. apart. A candle is placed halfway between them. Show by means of a diagram the position of the first three images seen in each mirror. Tell the distance of each of these images from the nearer mirror, and state the method by which you have determined these distances. [The course of the rays of light need not be shown in the diagram.]
8. Galvanic cell A has an electromotive force of 2 volts and a resistance of 6 ohms. Cell B has an electromotive force of 1 volt and a resistance of 3 ohms. How strong a current will each cell, used alone, send through a wire so short that its resistance may be neglected?
How strong will the current be if the two cells are joined in series, both working in the same direction '^,
Which cell will prevail, and how strong will be the current, if the two cells are joined in such a way as to oppose each other? [The electromotive forces and resistances would be likely to change in the course of such experiments. But this fact is to be disregarded.]
Admission, (i) 1895.
PHYSICAL SCIENCE 1 35
PHYSICAL SCIENCE ALTERNATIVE H: EXPERIMENTAL PHYSICS
[Omit three questions.]
1. In a hydraulic (or hydrostatic) press the area of the small piston face is 1 sq. in. and that of the large piston face is 50 sq. in.
(a) If a force of 50 lbs. is applied to the small piston, how great is the force exerted upon the large piston, provided there be no friction? .
(d) How much work is done upon the small piston while it moves 6 in., and how much is done at the same time upon the large piston? Name the unit in which the work is reckoned.
2. (\emph{a}) Mention two of the most dense and two of the least dense solids that you know of.
(b) A cu. ft. of water weighs 62.4 lbs. A certain block of wood floats in water with ^ of its volume above the surface. What is the density of this body in the foot-pound system 1
3. A cord is fastened at each end, and a weight is suspended from it at a certain point where the cord bends at a right angle. The pull exerted at one end of the cord is 3 lbs. and at the other end 4 lbs. How heavy is the suspended weight? [The weight of the cord is neglected.]
4. Describe your laboratory experiments upon the collision of ivory balls, and state the general law to which you were led by means of them.
5. (\emph{a}) Define the €Uw-point. At what time of the year and in what kind of weather is the dew-point very high? Under what conditions is it especially low?
(b) What is meant by the phrase, the mechanical equivalent of heat?
6. The velocity of sound being 340 m. per second, how long should a tube closed at one end be, in order that its resonance
may reinforce the sound from a tuning fork making 440 complete, that is, double, vibrations per second?
7. An object is placed near a double convex lens and then is moved outward from the lens along the principal axis until it passes far beyond the principal focus. Illustrate by means of a diagram the successive changes in position of the image caused by the movement of the object, showing the position of the principal axis and the principal focus, and marking several positions of the object as $O_j$, $O_2$, etc., and the corresponding images by $I_1$, $I_2$, etc.
8. (\emph{a}) Trace the lines of magnetic force in the neighborhood of a galvanometer coil in which a current of electricity is flowing from south to north above and from north to south below, taking into account the earth's magnetic force, and showing the resultant effect of combining these with the force due to the current.
(b) A certain galvanic battery sends a very weak current through a small external resistance. What experiments could you make to determine whether the weakness of the current is due to very small electromotive force or to any large internal resistance in the battery?
Admission, (i) 1896.
PHYSICAL SCIENCE ALTERNATIVE H: EXPERIMENTAL PHYSICS
[Omit any five questions.]
1. The law for the stiffness of a beam is :
$$\mbox{\emph{Stiffness is proportional to }} \frac{(width)\ \times\ (thickness)^3}{(length)^3}$$
If a beam 10 ft. long, 2 in. through in one direction and 4 in. through in the other direction, bends 1 in. under a force of 500 lbs. applied parallel to its 4-in. dimension, how far would it bend under the same force applied parallel to its 2-in. dimension?
2. A water tank 8 ft. deep, standing some distance above the
PHYSICAL SCIENCE 1 37*
ground, closed everywhere except at the top, is to be emptied. The only means of emptying it is a flexible tube.
(a) What is the most convenient way of using the tube and how could it be set into operation?
(d) How long must the tube be to empty the tank completely?
3. A certain body weighing 1 50 lbs. will just float in sea-water of specific gravity 1 .026. How great is the force, in addition to the buoyant force of the water, that would keep this body from sinking in fresh water?
4. A ladder 10 ft. long and weighing 20 lbs. leans against a vertical wall at an angle of 45°. The center of gravity of the ladder is at the center of its length. The wall is smooth, so that the force exerted where the ladder touches it is horizontal. Resolve the force exerted at the other end of the ladder into two forces, one vertical and one horizontal, and show how great each is.
[\emph{Suggestion: } Make use in this problem of your knowledge of the moments of couples.]
6. A mass of 2 lbs., moving with a velocity of 25 ft. per second, meets squarely a ball weighing 4 oz., moving in the opposite direction with a velocity of 80 ft. per second. After the collision the mass rebounds with a velocity of 2 ft. per second. What velocity has the ball just after the collision?
6. (\emph{a}) Show how a thermometer may be constructed without use of any liquid.
(b) Name, if you can, any liquids that boil at a lower temperature than water ; any that boil at a higher temperature than water.
7. How many inches of rainfall at a temperature of 10° C. would be sufl&cient to melt a layer of snow 1 ft. thick, of specific gravity 0.2, taken at 0° C.?
(" An inch of rainfall " means enough rain to make a layer of water 1 in. thick over the whole region of fall.)
8. Describe carefully some method which you have used for measuring the velocity of sound.
9. (\emph{a}) Define the principal focus of a lens.
(b) In what position with respect to this point is the object placed in the use of a simple magnifying glass?
(c) In what position with respect to the principal focus of the object-glass is the object placed in the use of a microscope?
10. A battery consisting of 3 cells in series, each having an electromotive force of 1 volt and an internal resistance of 2 ohms, is joined in circuit with another battery consisting of 3 cells in series, each having an electromotive force of 2 volts and an internal resistance of 1 ohm. In connecting the two batteries, like poles are joined together. There is no external resistance. How strong will the current through the circuit be?
Admission, (i) 1897.
BOTH MSTHODS EXPERIMENTAL PHYSICS
[Omit any three questions.]
1. How could you find the volume of an irregular lump of coal?
2. A ladder 20 ft. long and weighing 30 lbs., its center of gravity being at its middle point, stands upon the ground and leans at an angle of 45° against a smooth vertical wall. In this case the force exerted by the wall against the ladder is horizontal. How great is it.?
[\emph{Suggestion: } Calculate the moments with respect to the point where the ladder touches the ground.]
3. Explain with a diagram the principle of the reservoir of a student lamp.
4. Give reasons for believing that sound is a wave motion of the air or of some other medium transmitting it.
5. Describe and explain a good method of freeing a groundglass stopper which sticks hard in the mouth of a glass bottle.
EXPERIMENTAL PHYSICS 1 39
Describe with a diagram and explain the device used in the balance of a watch to prevent the rate of the watch from changing with change of temperature.
6. If the latent heat of melting of ice is 80, how many g. of ice at 0° must be put into 1000 g. of water at 30° C. in order that the final temperature of the whole may be 10° C.?
7. Represent a concave mirror by means of the arc of a circle drawn on y6ur paper. Mark the position of the center of curvature and of the principal focus. Draw the principal axis, and mark on it the position of a point the image of which would lie outside the center of curvature. Mark another point the image of which would be virtual.
8. A galvanic battery consists of 12 cells, each having an electromotive force of 2 volts and a resistance of 1 ohm. How great a current will this battery send through an external resistance of 26 ohms, if the cells are arranged in two sets of 6 cells each in series, and these two sets are joined together in multiple?
Admission, (i) 1898.
BOTH METHODS EXPERIMENTAL PHYSICS
[Candidates must hand in their notebooks at the time of the laboratory examination. Answer six questions.]
1. A block of wood having a volume of 200 c.c. and a specific gravity 0.5 is used as a float to support a ball of lead that weighs 50 g. when under water. How much of the block will be beneath the surface of the water, the lead being attached to its under surface?
2. A cylindrical tube open at both ends is pushed down vertically into mercury until only 10 cm. of its length remains above the surface. This end is then closed air-tight, and the tube is raised until the air confined in its top has expanded to 30 cm. What is the difference in level of the surface of the mercury inside and outside of the tube, the barometric pressure at the time being 75 cm. of mercury?
3. Two forces, one of 20 lbs. acting north, the other of 10 lbs. acting west, are applied at one point of a body. Two other forces, one acting south and the other acting east, are set to balance the first two, the point of application of the second pair being different from that of the first pair. Make a diagram to scale, showing how the whole may be arranged, and state the magnitude of each of the second pair of forces.
4. Make a diagram of a converging lens, drawing the principal axis and marking with a letter P the principal focus. Mark with an A the position of an object the image of which, as formed by the lens, will be real. Mark with a B the position of an object the image of which will be virtual. Will the image of the object at A be upright or inverted? larger than the object or smaller? Will the image of the object at B be upright or inverted? larger than the object or smaller?
5. If 200 g. of a metal of specific heat o.i are dropped into 50 g. of a liquid of specific heat 0.8, the metal being at 100° and the liquid at 20°, what will be the resulting temperature of the mixture?
6. If the weight of air in a chimney 10 m. tall and 0.2 sq. m. in cross-section is 2.58 kg., when the temperature of the air is 20° C, how great is the weight of air in the same chimney when the temperature of the air is 100° C.?
7. Describe briefly the voltameter method and the galvanometer method of measuring an electric current.
8. Define an induced current of electricity and describe the construction of some practical machine or piece of apparatus in which an induced current is generated.
9. Make a diagram showing an electric door-bell system, with one battery arranged to work two bells independently whenever the proper knobs are pushed.
Admission, (i) 1899.
EXPERIMENTAL PHYSICS I4I
BOTH METHODS EXPERIMENTAL PHYSICS
[Omit four of the following questions.]
1. Air is forced into a bicycle tire by means of a pump 4 sq. cm. in cross-section (inside) and having a stroke 1 5 cm. long. The tire is at first entirely empty, but at last contains 600 c.c. of air at a pressure five times as great as the atmospheric pressure. How many strokes of the pump, working perfectly, have been needed to produce this result?
2. A body weighing 60 lbs. rests on an incline, the length of which is 10 ft., the height 6 ft., and the horizontal base 8 ft.
(a) How great a force parallel to the incline is required to keep the body moving up the incline with a uniform velocity, there being no friction? ,
(d) The pressure of the body against the incline is 48 lbs. If the coefficient of friction is 0.2, how great is the force required to draw the body with uniform velocity up the incline?
3. Define the dyne and the erg. A force of 15 dynes acting for 8 seconds imparts to a certain free body a final velocity of 10 cm. per second.
(a) How great is the mass of the body?
(b) How much work does the force do on the body in giving it the velocity mentioned?
4. Define radiant energy and spectrum analysis. Mention any example in which spectrum analysis has been used.
5. If the mechanical equivalent of heat, on the basis of pounds, feet, and degrees Centigrade is 1400, how far must a body, the specific heat of which is 0.2, fall in order that the heat generated by its fall may be great enough to raise the temperature of the body 6° C?
6. Tell the vibration frequency of the notes E and G in the gamut which begins with 128 vibrations for the lower C.
7. What sort of eyeglasses should near-sighted^ or shortsighted, persons wear? Illustrate your answer by means of a diagram;
8. A candle flame is 6 ft. from a wall ; a lens is between the flame and the wall, 5 ft. from the latter. A distinct image of the flame is formed on the wall.
(a) In what other position may the lens be placed, that a distinct image may be formed on the wall t
(b) How will the lengths of the two images compare, and which will be the longer?
9. Define the following terms relating to dynamos or electric motors : armature, series wound, shunt wound.
10. How strong a current will a cell having an electromotive force of 2 volts and an internal resistance of 1 ohm send through an external circuit consisting of a i-ohm resistance and a 2-ohm resistance joined in multiple (or parallel) circuit?
June, 1900.