HEAT
1. Reduce 8®, 20®, 70®, — 40°, and 10° C. to the Fahrenheit scale.
2. Reduce 8°, 40°, I8o^ — Io^ and 36° F. to the Centigrade scale.
3. What must be the temperature of a liquid so that both the Fahrenheit and Centigrade thermometers shall read the same when immersed in it?
4. Absolute zero is 273° below zero on the Centigrade scale. What is this temperature on the Fahrenheit scale?
5. Iron melts at 1200° C. What is the melting point on the Fahrenheit scale?
6. The average difference in temperature between two places is 60® F. How much would this be on the Centigrade scale?
7. Lead expands .000028 of its length for every degree Centigrade. How much would it expand for a degree Fahrenheit?
8. The coefficient of linear expansion of iron is .000011. How much must an iron rod 40 ft. long be heated to expand I in.?
9. At 20° C. a steel meter rod is found to be 100.01654 cm. long. At what temperature will it be correct?
10. The brass pendulum of a clock is 1 m. long when the temperature is 10° C. (\emph{a}) What will be its length at 0° C.? (\emph{b}) at 25^ C.?
11. A wire 10 ft. long at o° C, heated 90**, increases its length .3 1 in. What is its coefficient of linear expansion?
12. An iron rod 250 cm. long at 10° C. is .09 cm. longer at 40° C. What will be its length at 60° C.?
13. How much room must a steel pipe 100 ft. long at 20° C. have left for expansion if its maximum temperature is to be 100° C.?
14. A brass rod is 1 m. long at 0° C. How long will it beat 75° C?
16. An iron rod is 105.4 cm. long at 100° C. What is its length at o° C.?
16. A rod 10 m. long is found to have expanded .95 cm. when heated from 0° to 50° C. What part of its length at 60° C. would it expand if heated from 60° to 70° F.?
17. The barometer at 0° C. stands at 750 mm. At what height will it stand if the temperature rises to 30° C, the pressure remaining the same? Disregard the expansion of the glass and scale.
18. The steel rails in a certain road are 30 ft. long at 0° C, and the difference between the extremes of summer and winter temperature is 117° F. What will be the difference in length of the rails between these two temperatures?
19. On a day when the temperature of the air is 25° C, a 40-yd. track is measured off with a lo-ft. steel tape correct at 0° C. What will be the error in the length of the track?
20. What is the length of a steel rod which increases 1 in, when heated 200° C?
21. A brass tape 30 m. in length at 0° C. was graduated to jj^ of a cm. When the thermometer was 30° C. it was used to measure a kilometer. How much was the error in measurement?
22. A steel locomotive driving-wheel is 6 ft. in diameter at 0° C. How much farther will it go per revolution when its temperature is 25° C.?
23. Two rods, one of zinc and the other of platinum, which at zero measure exactly 1 m. each, are heated to 90° C. What part of the length of the platinum rod at 90° C. will be the gain in length of the zinc rod?
24. Two rods, each 2 m. long at 0° C, one brass and the other iron, are laid side by side and joined at one end. Will there be any difference in their lengths when heated to 25° C.? If so, how much.?
26. A platinum rod and a brass rod each measure 1 m. at o° C. How much must they be heated in order to have the brass rod 2 mm. longer than the platinum rod?
26. If at 0° C. an iron rod measures 201 cm. and a brass rod 200 cm., to what temperature must they be raised to be of the same length? What will their common length be?
27. If a 4-m. brass wire stretches 1 cm. with a force of 2 kgm., how much weight could be lifted by cooling this wire from 120° C. to o°C.?
28. In a linear expansion apparatus of the lever form the long arm of the pointer is 36 cm. long, and the short arm .4 cm. The pointer rises 5.4 cm. when a rod in the apparatus is raised from 25° C. to 100° C. What is the coefficient of linear expansion of the rod if its length is 75 cm.?
29. Find the coefficient of linear expansion of a rod from the following data taken from an experiment with the expansion apparatus :
Length of rod ' 60 cm.
Length of long arm of lever 24 "
Length of short arm of lever 2.5 '*
Height of long arm on scale before heating . 5 " Height of long arm on scale after heating . .5.48 "
Temperature of rod before heating .... 20° C.
Temperature of rod after heating 100° C.
30. A plate-glass window is 10 ft. by 12 ft. How much will it change in area if its temperature changes from 0° C. t0 2S**C.?
31. An iron rod i.ooi cm. in diameter of cross-section at 0° C. is to have a brass ring, which at the same temperature is 1 cm. in internal diameter, shrunk to it. To what temperature must they both be heated that the ring may be placed on the rod?
32. A glass liter flask is filled with mercury at o° C. and then heated to 100° C. How much mercury runs out? Disregard the expansion of flask.
33. The sp. gr. of mercury at 0° C. is 13.6. What is it at 300° C.?
34. How much mercury must be below the zero point of a Centigrade thermometer when plunged into a freezing mixture, if the diameter of the tube is ^^^ of a mm., in order that the length of a degree may be 2 mm.? Disregard the expansion of the glass.
35. A cubical block of gold was found at 0° C. to contain 8000 c.c. What will be the contents at 100° C.? How much more will it weigh?
36. An aluminum cylinder is 10 cm. long and 3 cm. in diameter measured at 0° C. (\emph{a}) What is its volume at 100° C. .^ (\emph{b}) If its density is 2.7 at the lower temperature, what is it at the higher?
37. A glass beaker measures 10 cm. deep and 5 cm. in diameter at o^ C. How much water will it hold at 100° C.? Disregard the expansion of the water.
38. A liter flask is correct in cubical contents at 0° C, and the density of mercury is 13.6 g. per c.c. at o° C. How many grams of mercury will the flask hold at 50° C.?
39. A cubical box of copper 10 cm. on a side is filled with mercury, and then heated until 1 c.c. of mercury runs out. How many degrees Centigrade is its temperature raised?
40. How much longer must an iron rod be than one of copper in order that the two may expand the same amount for every degree C.?
41. A solid glass globe 10 cm. in diameter when measured at 0° C, and weighing 1800 g., was placed in water and the temperature raised to 100° C. Supposing that the sp. gr. of the water remains i, what will be the apparent weight of the globe when thus immersed?
42. There is a zinc rod 5 m. long and 20 cm. square at o° C, which is to be covered by a coating 1 mm. thick. How many more sq. mm. of coating will be required when the temperature is 30° C. than when it is 0° C.?
43. When 200 g. of brass at 100° C. are poured into 100 g. of water at 40^.5 C, the temperature of the water is raised to 50° C. What is the specific heat of brass?
44. Five hundred g. of a certain substance at 100° C. are poured into 140 g. of water at 20° C, raising its temperature to 40° C. What is the sp. ht. of the substance?
45. Ten g. of iron at 100° C. are plunged into 9 g. of water at 10° C, raising the temperature of the water thereby to 20° C. What is the sp. ht. of the iron? '
46. Ten g. of nickel are heated to 100° C, and placed in 5 g. of water at 0° C. What is the resulting temperature?
47. How much brass at 100° C. must be turned into 20 g. of snow at 0° C. to melt the snow and raise the temperature to 40° C.?
48. How much heat will it take to raise 20 g. of silver from 0° C. to 60° C.?
49. How much heat will be given out by 8 g. of zinc in cooling from 75* to 5° C.?
50. Fifty g. of copper at 100° C. are plunged into 100 g. of water at 0° C. What will be the resulting temperature?
51. One hundred g. of lead at 100° C. are put into 200 g. of alcohol at 0° C. What will be the resulting temperature?
52. One hundred g. of water at 80° C. are thoroughly mixed with 500 g. of mercury at 0° C. What is the temperature of the mixture?
53. How much ether at 30° C. must be mixed with 10 g. of water at 6° C. that the resulting temperature may be 10° C?
54. A pound of sulphur at 0° C. is placed in a pound of water at 122° F. What will be the resulting temperature?
56. Fifty g. of mercury at 100° C. are mixed with 50 g. of water at 0° C. What is the resulting temperature?
56. How much mercury at 100° C. must be mixed with 40 g. of water at 10° C. that the resulting temperature may be 25° C.?
57. A rod composed of platinum and silver and weighing i6 g. is heated to loo° C, and dropped into lo g. of water at 14° C, the temperature of the water being raised thereby to 20° C. How much platinum in the rod?
58. How much tin at 100° C. must be put into 100 g. of mercury at 10° C. to raise the temperature to 20° C.?
59. One hundred g. of lead at 0° C. are placed in 50 g. of water at 212° F. The water is found to lose 2° C. when the lead is taken out. How much has the temperature of the lead increased?
60. A bar of iron weighing 5 lbs. at 0° C. is placed in 2 lbs. of water at 100° C, and after remaining for a few minutes is taken out, when its temperature is found to have risen to 40° C. What is the temperature of the water .?
61. Find the latent heat of melting ice from the following data :
Weight of calorimeter 60 g.
Weight of calorimeter and water 460 g.
Temperature of water 38° C.
Temperature of the mixture 5° C.
Weight of calorimeter, water, and ice . . . . 618 g.
Specific heat of calorimeter .1
62. Find the latent heat of melting ice from the following data:
Weight of calorimeter 80 g.
Weight of calorimeter and alcohol 53° g-
Temperature of alcohol 30° C.
Temperature of the mixture 8° C.
Weight of the calorimeter, alcohol, and ice . . 604 g.
Specific heat of alcohol .64
Specific heat of calorimeter .1
No chemical action. ^
63. How much heat will it take to change 6 g. of ice at 0° C. to water at 40° C.?
64. On a cold night in the winter a fanner fills a cubical tank in his cellar with water. In the morning he finds that the water has just begun to freeze. How much heat has been given out to the cellar? The tank is ^ m. on a side. The temperature of the water when put in was 70° C.
65. Just what will occur if 1000 calories be applied to 20 g. of ice at o° C.?
66. How many inches of rain at 41** F. must fall in order to melt J in. of ice at o° C.?
67. How much heat will it take to melt a mass of 19 g. of zinc taken at o° C.?
68. How much heat will be used in raising 50 g. of iron from o° C. to the melting point and just melting it?
69. How much water at 100° C. must be mixed with 10 g. of ice at - 10° C. that the resulting temperature may be io*C..>
70. How many grams of iron at 90° C. will be required to change 8 g. of ice at 0° C. into water at 25° C.?
71. Fifty g. of melted tin at 235° C. are poured into 40 g. of water at 0° C. What will be the^ resulting temperature?
72. Four hundred g. of melted lead at 330° C. are poured into 50 g. of ice at 0° C. What will be the result if there is no loss of heat by radiation?
73. Assume that mixing has no effect on the thermal capacities of oil of turpentine and alcohol. What will be the resulting temperature if 5 g. of ice at o° C. are dropped into a lo-g. mixture of equal parts of these substances at 70° C.?
74. Ten lbs. of zinc at 80° C. are poured into a pound mixture of snow and water, raising the temperature thereby to 20° C. How much snow in the mixture?
75. Into a 1 lb. mixture of snow and water are turned 2 lbs. of water at 29° C. The resulting temperature is found to be 2° C. How much snow was in the mixture?
76. How much melted zinc at 420° C. must be poured into JOG g. of mercury at 0° C. to raise its temperature to 80° C.?
77. Find the latent heat of vaporization of steam from the following data :
Weight of calorimeter 300 g.
Weight of calorimeter and water 750 g.
Weight of calorimeter, water, and steam 790.25 g.
Temperature of the water 6° C.
Temperature of the steam 100° C.
Temperature of the mixture 55° C.
Specific heat of calorimeter .1
78. Find the latent heat of vaporization of steam from the following data :
Weight of calorimeter 200 g.
Weight of calorimeter and turpentine 820 g.
Temperature of turpentine 4° C.
Temperature of the mixture 32° C.
Temperature of the steam 100° C.
Weight of calorimeter, turpentine, and steam 834.4 g.
Specific heat of turpentine .47
Specific heat of calorimeter .1
79. Eight g. of steam at 200° C. are cooled to 0° C. How much heat is given out?
80. Ten g. of steam at 100° C. are cooled to 41° F. How much heat is given out?
81. One lb. of water at 10° C. is the result of cooling the same mass of steam from 110° C. How much snow at — 10° C. would be melted by the heat thus given out?
82. How many calories of heat will be needed to change 8 g. of ice at — 40° F. to steam at 500° F.?
83. Four g. of ice at —10° C. are changed into steam at 200° C. How much heat is used?
84. How much heat will it take to melt 5 g. of ice at 0° C. and change it into steam at 100° C.?
85. How many grams of ice at — 10° C. will be turned into water at 50° C. by 10 g. of steam at 200° C.?
86. How much heat would be required to change 20 g. of ice at — 15° C. to steam at 765° C.? The pressure of the steam is constant.
87. One hundred g. of steam at atmospheric pressure are condensed by passing into a 500-g. mixture of ice and water at 0° C. The final temperature is 70° C. How many grams of ice were there in the mixture?
88. How many units of heat will be required to evaporate 6 g. of alcohol at 0° C.?
89. A 500-g. mass of iron and copper turnings, taken in equal parts and at a temperature of 1000° C, is put into 100 g. of alcohol at — 10° C. What is the result? Specific heat of alcohol vapor is .45.
90. How much liquid lead at 330° C. must be put into 10 g. of oil of turpentine at 1 0° C. to evaporate it?
91. How much melted zinc at 420° C. must be placed in 20 g. of alcohol at 0° C. to evaporate it?
92. How much heat will be given off by 100 g. of sulphur vapor in cooling from 448° C. to 1 o° C.?
93. How much melted tin at 230° C. must be poured into 10 g. of oil of turpentine at 10° C. to evaporate it.**
94. To how many ft. lbs. of energy are 8 calories of heat equivalent? ,
95. An iron anvil weighing 100 lbs. is struck by a hammer weighing 10 lbs., and moving at the rate of 30 ft. per second. If all the energy is used in heating the anvil, how majiy degrees is its temperature raised?
96. On a day when the temperature is 20° C, 5 cu. ft. of water at 20° C. are poured from a certain height into 10 cu. ft. of water at o° C. The temperature of the water after mixing is found to be 7° C. From what height did the water fall? A cu. ft. of water weighs 62.5 lbs. No energy was lost.
97. How far above the surface of the earth must a cake of ice be carried to be melted by the impact of its fall, if all the energy is changed into heat?
98. A body weighing 10 kg. and moving along the ice with a velocity of 5 m. per second is stopped by the friction of the surface. If all the energy is changed into heat, which acts upon the surface of the ice, how much ice is melted?
99. How many inches of rain at 10° C. must fall to melt ^ in. of ice at o° C. ^ Consider the sp. gr. of ice 1.
100. How many inches of rain at 10° C. must fall to melt J in. of ice at 0° C.? Consider the sp. gr. of ice .9.