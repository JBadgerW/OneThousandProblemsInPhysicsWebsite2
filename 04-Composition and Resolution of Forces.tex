COMPOSITION AND RESOLUTION OF FORCES
1. What is the tension on a wire whose ends are attached, one to a hook in the ceiling of a room, the other to a 25-lb. weight which it supports?
2. Eight men are engaged in a tug of war which results in a draw. Each man exerts a force of 300 lbs. What is the tension on that part of the rope between the two teams?
3. Find the magnitude and position of the resultant of two parallel forces of 16 and 24 lbs., respectively. They act in the same direction and their points of application are 5 ft. apart.
4. Two forces of 6 and 8 units act upon a body in lines which meet in a point and are at right angles. Find the magnitude of their resultant.
5. Resolve a force of 500 kg. acting N.E. into two forces, one acting E. and the other N.
6. Resolve a force of 100 lbs. into two forces acting at right angles to each other, one of which shall be twice as great as the other.
7. Resolve a force of 50 g. into two forces at right angles to each other, one of which shall be three times as great as the other.
8. How great must be the force that can be resolved into two forces, the smaller of which, 50 lbs., acts N., and the larger, 80 lbs., acts E. In what direction must it act?
9. Two forces which are in the ratio of 2 : 3 act perpendicularly to each other upon a point and produce a resultant force of 26 lbs. Determine the value of these forces.
10. What force can be resolved into two components at right angles to each other, one of which shall be 20 and the other 60 lbs.?
11. Resolve a force of 20 lbs. into two components acting at an angle of 60° with each other.
12. Resolve a force of 30 lbs. acting vertically into two components, one of which acts at an angle of 30° to the vertical and the other at an angle of 45° to the vertical.
13. There is a force of 1000 g. acting S. This is kept in equilibrium by two forces, one acting N.E. and the other W. Determine by plotting the magnitude of each force.
14. The force £A of 10 lbs. acts N. upon the point A. The force C^ of 15 lbs. makes an angle of 30° with BA to the K, and DA of 20 lbs. makes an angle of 60° to the W. Find the resultant.
16. The wind is blowing to the S.W. with an actual velocity of 20 miles per hour. Find its southerly and westerly components.
16. Forces of 10 lbs. acting N. ; of 5 lbs. acting N.E. ; of 20 lbs. acting E. ; and of 15 lbs. acting S.E. impinge upon a body, (\emph{a}) In what approximate direction will the body move? (\emph{b}) By what resultant force is it acted upon?
17. A balloon rises 500 ft. per minute and at the same time is blown along by the wind at the rate of 300 ft. per minute. What is its rate of motion?
18. At what angle can two equal forces act upon each other so that equilibrium is produced?
19. The resultant of two forces is 20. One of the forces is 12, and the other is inclined to it at an angle of 90°. What is the second force.-*
20. Two forces of 10 lbs. each act at an angle of 120** to each other. What is the resultant force?
21. A railway car is moving at the rate of 30 miles per hour. A ball is moved directly across the car at the same rate, (\emph{a}) How fast does the ball move relatively to the ground? (\emph{b}) At what angle to the direction of motion of the car?
22. Two boys strike a football at the same instant, one with a force of 25 lbs. acting N., and the other with a force of 30 lbs. acting E. What direction will the ball take?
23. The southerly component of the wind is 10 miles per hour and the westerly component is 15 miles per hour. Find the direction and velocity of the wind.
24. A body is moved E. with a constant velocity of 40 ft. per second and N. with a constant velocity of 60 ft. per second. What is its actual rate of motion in a straight line?
25. A ship* borne N. by the wind at the rate of 10 miles per hour strikes a current flowing N.E. at the rate of 6 miles per hour, (\emph{a}) What direction will the ship take? (\emph{b}) How far will it travel in one hour?
26. A boat is rowed at right angles to the course of a river twice as fast as the river flows. It makes the opposite shore 1 J miles below the starting point. What is the breadth of the river?
27. A man rowing a boat across a river 3 miles wide, at the rate of 4 miles per hour, is carried down by the current at the rate of 2.5 miles per hour, (\emph{a}) How far from the starting place will he be when he reaches the other side? (\emph{b}) How far below the place directly opposite the starting point?
28. A rod extending N. and S. has 2 cm. from the north end a force of 20 g. acting E. ; 4 cm. from the north end a force of 15 g. acting W. ; 6 cm. from the north end a force of 10 g. acting E. ; and 8 cm. from the north end a force of 5 g. acting W. (\emph{a}) Where must a force be applied to produce equilibrium? (\emph{b}) How great must it be?
29. A weightless rod, AB, is 12 cm. long and has acting upon it 5 forces; one at end A of 10 g. acting down; one at end B of 6 g. acting down ; 3 cm. from A a force of 16 g. acting up ; 7 cm. from A a force of 12 g. acting up; and 2 cm. from B a force of 5 g. acting down, (\emph{a}) What is the value of the equilibrating force? (\emph{b}) How far from B must it be placed?
30. A board 3 ft. square is so placed that its sides extend in the directions of the four cardinal points of the compass. One ft. west of the east end of the north side a force of 10 lbs. acts north, and 1 ft. east of the west end of the south side a force of 10 lbs. acts south, (\emph{a}) Where can two other forces be placed so as to keep the board in equilibrium? (\emph{b}) How great must they be?
31. A board 2 ft. squar6 so placed that its sides extend N., S., E., and W. has attached to the N.W. corner a force of 100 lbs. acting N., and from the center of the south side a force of the same amount acting south, (\emph{a}) In what direction must forces act upon two of the corners to produce equilibrium? (\emph{b}) Upon what corners must they act? (\emph{c}) How great must they be?
32. A horse is attached to a sleigh so that the traces make an angle of 30° with the surface of the snow. Supposing the sleigh offers a resistance of 500 lbs. to horizontal motion, how much must the horse pull to move it?
33. There is a hill the surface of which is inclined at an angle of 45** to the horizon. A sled weighing 100 lbs. rests upon this. How much of the weight of the sled will tend to produce motion down the hill?
34. A sled weighing 50 lbs. rests on the side of a hill rising 1 ft in 30 along the incline. What force acting parallel to the surface of the hill will keep the sled from sliding down the hill? Disregard friction.
36. A man drags a weight by a rope passing over his shoulder and making an angle of 45° with the surface of the ground. The tension on the rope is 75 lbs. If the friction is not considered, how much force acting parallel to the ground would it take to move the weight?
36. The horses attached to a car are pulling at an angle of 45° to the track with the force of 1 ton. (\emph{a}) How much of this force is used in pulling the car along the track? (\emph{b}) How much in pressing the wheels against the side of the track?
37. A pair of horses moving on a level against a force of 1800 lbs. pull at an angle of 90^ to each other. One pulls J more than the other. How much does each pull 1
38. Resolve the force 48 into two forces making angles of 45° with the given force on either side of it.
39. An inclined plane rises 1 ft. in 8 of horizontal distance. What force acting up the plane and parallel to it will just sustain a mass of 168 lbs. on the incline?
40. A body weighing 10 lbs. rests upon an inclined plane which rises 3 ft. in 4 of horizontal distance. What is the pressure normal to the plane?
41. What must be the breaking strength of a board which, when inclined at an angle of 45° with the horizon, just sustains a lo-lb. load moved by a force acting parallel to the surface of the board?
42. A weight of 500 lbs. rests upon an inclined plane 20 ft. long and 8 ft. high. What must be the force of friction to keep it from sliding down the plane?
43. There is a uniform gate 4 ft. long and 3 ft. high which weighs 50 lbs. It is attached to a post by hinges at the top and bottom, (\emph{a}) What horizontal pull or pressure must each of these hinges overcome? (\emph{b}) What vertical pull? The load is divided equally between the hinges.
44. A uniformly thick board in the shape of an equilateral triangle 1 m. on a side and weighing 5 kg. is suspended with one of its sides vertical by means of hinges placed at the extremities of this side. What is the horizontal pull or pressure on each of these hinges?
45. A mass of 1 kg. is supported by a cord passing over a nail so that the two 'parts of the cord make an angle of 90° with each other. Find the tension on the cord.
46. A mass of 50 g. is attached to the center of a string passing over two frictionless pulleys so arranged that the angle between the two parts of the string on each side of the mass is 45°. How many grams must be attached to each end of the string after passing over the pulleys to keep the 50-g. mass in equilibrium?
47. A cord supporting a picture weighing 10 lbs. passes over a knob so that there is an angle of 60° between the two parts of the cord. How many pounds must the cord be able to support in order to hold the picture?
48. A rope 1 2 ft. long is attached to each end of a horizontal beam 8 ft. long. From the center of the rope is suspended a mass of 100 lbs. What is the tension on the rope?
49. A board 3 ft. square and weighing 1 o lbs. is to be supported from a beam by two equal cords attached to the extremities of the upper edge of the board, (\emph{a}) Will the cords need to be stronger if they are attached to the same point on the beam than if they extend vertically from their points of attachment on the board to points on the beam? (\emph{d}) If so, how much stronger? The upper edge of the board is to be 3 ft. frpm the beam.
50. A picture weighing 25 lbs. is hung by a cord passing over a nail, the two parts of the cord making an angle of 120'' with each other. What is the tension on the cord?
51. Two strings 6 and 10 ft. long, attached to a horizontal beam, meeting at a point and making an angle of 60'', support a 50-lb. weight. What is the tension on each string?
52. A fish caught by a pole and line pulls with a force of 5 lbs. The inclination between the pole and the line is 45''. How many lbs. pull at right angles to its length must the pole be able to bear to keep it from breaking?
53. There is a rod 3 ft. long projecting horizontally from the side of a house, but not fastened to it. Four ft. vertically above the point where the rod joins the house there is a rope one end of which is attached to the house and the other to the outer end of the rod. From the point where the rope joins the rod is suspended a weight of 8 lbs. Find (\emph{a}) the horizontal pressure exerted by the rod on the wall ; (\emph{b}) the tension on the rope. Consider the rod weightless.
54. A horizontal rod 10 cm. long is pivoted at one end and has attached to it 2 cm. from the pivot a weight of 50 g. How great a force acting on the free end at an angle of 45° to the rod will be required to keep the rod in a horizontal position?
55. A uniform beam 6 ft. long, weighing 50 lbs., has a rope 10 ft. long attached to its center. If the free end of the rope is attached to the wall 15 ft. above the ground, and the beam is placed in a horizontal position at right angles to the wall, with one end resting against the wall, what will be the tension on the rope?
56. In the previous problem what will be the pressure of the beam against the wall?
67. A beam lo ft. long and weighing loo lbs. leans against the side of a house, the top of the beam being 6 ft. above the ground. If its center of gravity is 6 ft. from the lower end of the beam, what will be the horizontal force exerted by the house? Disregard friction between beam and wall.
58. A ladder 20 ft. long and weighing 50 lbs. rests with one end against a smooth wall and the other end on the ground 8 ft. horizontally from the wall. The center of gravity of the ladder is 9 ft. from the base, (\emph{a}) What is its horizontal pressure against the wall? (\emph{b}) What is the horizontal force exerted by the ground?
69. A uniform beam 10 ft. long and weighing 50 lbs. leans against the side of a house, making an angle of 45** with it. Two ft. from the top of the beam is suspended a weight of 100 lbs. What is the horizontal pressure of the beam against the house? ' Disregard friction between house and beam.
60. A plumb bob weighing 5 lbs. is pushed aside by a horizontal force until the cord supporting it makes an angle of 60° with the vertical line, (\emph{a}) How much does the cord now sustain? (\emph{b}) How great is the horizontal force?
61. A uniform rod 1 m. long knd weighing 1 kg. has one end resting upon the ground and is kept at an angle of 45** with the surface by a cord which is attached to the opposite end, and makes an angle of 90° with the rod. What is the tension on the cord?
62. The beam of a derrick makes an angle of 45° with the horizon, and the rope supporting the beam an angle of 45** with the beam. From the upper end of the beam is suspended a weight of 500 lbs. (\emph{a}) What is the pressure on the beam? (\emph{b}) What is the tension on the rope? Disregard the weight of the beam.
63. A square lot has a post at each corner. A rope is passed around these and tightened so that the tension upon it is 50 lbs. What is the pressure on each post.^^
64. A uniform rod, pivoted at one end, hangs vertically. If a force equal to J the weight of the rod acts horizontally at the lower end, at what angle with the vertical will the rod be when in equilibrium?