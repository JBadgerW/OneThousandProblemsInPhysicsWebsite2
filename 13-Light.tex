LIGHT
1. The nearest fixed star is 25,000,000,000,000 miles from the earth. How long does it take light to come from this star to the earth?
2. What must be the relative areas of two planes in order that they may intercept the same amount of light from a luminous point, if one plane is 3 times as far as the other from the point?
3. If the diameter of the earth were twice as great, how much more of the sun's light would the earth intercept than it now does?
4. What part of the light given out by a lamp does a board 1 ft. square and 50 ft. distant from the lamp receive?
5. If the sun is 93,000,000 miles away, and the diameter of the earth is 8000 miles, how much of the light given off by the sun strikes the earth?
6. A body situated 100 ft. from a luminous point is moved 20 ft. nearer to the point. How much more light does it intercept than before?
7. The areas of two plane surfaces are to each other as 2:6. What will be the ratio of their distances from a luminous point if each of them intercepts the same amount of light?
8. In determining the illuminating power of a lamp by a Bunsen photometer the distance from the lamp to the screen was 100 cm. and from the screen to the candle 30 cm. What was the candle power of the lamp?
9. How many candles will be required to produce the same intensity of illumination at 2 m. distance that is produced by 1 candle at 30 cm. distance?
10. How many candles must be placed 100 cm. from a screen in order to illuminate it as much as two candles placed 1 2 cm. from the screen?
11. Two lights are distant from a screen 80 ft. and 12 ft. If the intensities of their illuminations of the screen are as 2:3, respectively, what are the relative intensities of the two lights?
12. An 8-candle power lamp and a candle are placed 5 m. apart. How far from the lamp in a straight line joining the flames must a screen be placed that it may be equally illuminated by each of them?
13. The candle powers of two lamps are to each other as 3 : 4 and their distances from a screen as 2 : 3. What are the relative intensities of their illumination of the screen?
14. The diameters of two spherical bodies are to each other as 2 : 3 and their distances from a luminous point as 3 : 4. What is the relative amount of light intercepted by each?
16. An 8-candle power lamp is 10 ft. away from a standard candle. On a straight line which is of indefinite length and passes through the lamp and candle, select two points at either of which a screen will be equally illuminated by lamp and candle.
16. Two lights whose candle powers are 5 and 7 are placed 10 ft. apart. At what point between them must a screen be placed that the intensity of illumination of the side toward the 5-candle power light may be twice as great as that of the opposite side?
17. The intensities of two lights are as 4 : 9 and their distance apart is 20 ft. How far from the weaker light and on a line joining them must a screen be placed to be equally illuminated on each side?
18. The diameter of the moon is about 2160 miles and of the earth 8000 miles. The distance from the sun to the earth is 93,000,000 miles, and from the moon to the earth 240,000 miles. If the sun were a point, what tangential flat area on the earth would be covered by a total eclipse?
19. Using the data of the previous problem, what part approximately of the light emitted by the sun, considering the moon as a perfect reflector, reaches the earth on a night when the moon is full?
20. What is the length of the shadow of a tree 50 ft. tall when the sun is 45° above the horizon?
21. What is the height of a tree which casts a shadow 100 ft. long when a rod 5 ft. high casts a shadow 7 ft long?
22. A tree stands on the margin of a pool which is 30 ft. wide. On the ground 5 ft. back of the opposite margin of the pool stands a person whose eye is 5 ft. above the level of the pool. In this position the person can see the image of the tree just touching his margin of the pool. How high is the tree?
23. The image of a stake 8 ft. long and 10 ft. from a shutter is seen on a screen 4 ft. from the shutter. The aperture in the shutter through which the light from the stake passes is minute. What is the size of the image?
24. The floor of a room 20 ft. wide and 10 ft. high is 2 ft. below the level of the surrounding ground. Through a small hole in the center of the side of the room the image of a man 6 ft. tall and standing at a distance of 25 ft. from the hole is thrown upon the opposite wall. How far from the floor will the head of the man appear?
26. What is the greatest angle at which the light from an object can strike a plane mirror and form an image?
26. The clock on a wall indicates 9.30. What time will it appear to indicate to a person seeing the reflection of the clock in a mirror on the opposite wall but too far away to see the figures distinctly?
27. A plane mirror lies upon a table and a pencil 6 in. long stands on one of the edges of the mirror. How long must the mirror be that the entire pencil may be seen reflected in it by a person whose eyes are lo in. horizontally from the edge of the mirror next him and 5 in. above the table?
28. A man standing in front of a mirror closes one eye and then places upon the mirror a piece of paper just large enough to cover the image of the closed eye from the sight of the open eye. He then closes his open eye and opens the other. Will he now be able to see the image of the closed eye?
29. A square plane mirror hangs in the center of one of the walls of a cubical room. What must be the size of the mirror that an observer with his eye in the center of the room may see the whole of the opposite wall reflected in it? [Deschanel.]
30. Prove that if an object is placed in front of a plane mirror and the mirror is moved parallel to itself, either toward or from the object, the image will move twice as far as the mirror.
31. A lamp is placed at the bottom of a vertical wall, and a plane mirror is to be placed at the top of the wall in such a position as to reflect the light from the lamp horizontally. What angle will the mirror make with the horizon?
32. Show by a drawing that a man cannot see himself at full length in a vertical mirror unless the mirror is at least 1 as long as he.
33. Prove that if a candle is placed in front of a vertical plane mirror and the mirror revolved 45° about a vertical axis the image will move through an arc of 90°.
34. A line 3 ft. long is drawn across a table ; and a mirror 4 in. long is placed with the center of its back over the center of the line, the surface being parallel to the line and at right angles to the surface of the table ; 5 in. from the end of the line and 3 in. from the line on the side facing the mirror a candle is placed. Will there be an image of the candle formed, and if so, where?
35. Two plane mirrors facing and parallel to each other are 6 cm. apart. An object is placed 2 cm. from one of the mirrors and between them. Calculate the distance from the mirror (\emph{a}) of the third image ; (\emph{b}) of the fifth image.
36. Show by a drawing the position of the images formed of an object placed midway between two plane mirrors whose surfaces make an angle of 45® with each other.
37. Two plane mirrors are placed at an angle of 45° to each other, and a candle is placed at a perpendicular distance of 10 cm. in front of each. At what distance back of the mirrors will the images appear after three reflections?
38. Show that the images formed by the third reflection of an object in each of two plane mirrors placed at an angle of 60° with each other lie at a common point back of the mirrors.
39. At what angle with each other must two plane mirrors be placed that there may be formed eleven images of an object which is 5 cm. from each mirror?
40. If the earth were a perfect sphere 8000 miles in diameter, how far away from a man at sea whose eyes are 15 ft above the level of the water will the horizon be as seen by him?
41. A picture 10 ft. square hangs on a wall. How small a mirror on the opposite wall 20 ft. away, the center of which is directly opposite the center of the picture, will reflect an image of the picture entire to a man whose eye is halfway between the picture and th^ mirror and on a level yri^h th^ir centers?
42. The minute spaces on the dial of a clock are 1 in. apart, and the hands are 2 in. from the face of the dial. When it is just 12 o'clock, it appears to be 2 minutes past 12 to a person standing at a certain position in front of the clock. When he walks 100 ft from his first position in a direction parallel to the face of the clock, the time appears to be 3 minutes before 12. If the circular arrangement of the spaces on the dial is not considered, what is the distance of the person from the clock? Neglect time required to walk 100 ft.
43. The radius of curvature of a concave mirror is 30 cm. How far from the mirror on the principal axis will the image of an object which is 50 cm. from the mirror appear?
44. If the radius of curvature of a concave mirror is 50 cm. and a candle is placed 40 cm. from the mirror, where will its image appear?
45. An arrow 6 in. long is placed 20 in. in front of a concave mirror whose radius of curvature is 2 ft. Where will the image appear and what will be its length?
46. In the previous problem, how long will the image be (\emph{a}) if the object is 5 ft. in front of the mirror? (\emph{d}) if it is 2 in. in front of the mirror?
47. What is the focal length of a concave mirror whose radius of curvature is 18 in.? Where will the image of a candle placed at the center of curvature appear?
48. What kind of an image will be formed by the mirror of the preceding problem if the object is placed 25 in. from the mirror? How large an image?
49. A real image formed by a concave mirror, the focal length of which is 20 cm., is twice as large as the object. Where are the object and image situated?
50. An image produced by a concave mirror is 8 times as large as the object. If the focal length of the mirror is lo in., where are the object and image situated?
51. A man stands with his eye exactly at the center of curvature of a concave spherical mirror. How many images of himself can he see?
52. When a candle is placed on the principal axis of a concave spherical mirror 50 cm. from the mirror, a real and inverted image is formed on a screen held at a greater distance from the mirror than the candle. If the image is 3 times as large as the object, what is the focal length of the mirror?
53. An object is placed 45 cm. in front of a concave spherical mirror the radius of curvature of which is 80 cm. What change will be made in the size and position of the image if the object is moved 10 cm. nearer the mirror?
54. A candle placed in the axis of a concave spherical mirror 20 cm. from the mirror shows a real image 50 cm. from the mirror. What is the radius of curvature of the mirror?
56. Prove that the image formed by a convex mirror is always smaller than the object.
56. If an object is placed in front of a convex mirror and at a distance equal to the focal length of the mirror, what will be the relative size of image and object?
57. If an object is placed at a distance in front of a convex mirror equal to the radius of curvature of the mirror, (\emph{a}) where will the image be? (\emph{d}) of what size?
58. An object is placed 10 cm. in front of a convex mirror the radius of curvature of which is 30 cm. (\emph{a}) Where is the image? (\emph{b}) Of what size?
59. If in the previous problem the object had been placed 20 cm. from the mirror, where would the image have appeared?
60. In Problem 58, what would have been (\emph{a}) the position of the image if the object had been placed 40 cm. from the mirror? (\emph{b}) the size?
61. Show by a diagram the refraction between two media when the angle of incidence is (\emph{a}) 60° and the angle of refraction is 30° ; (\emph{b}) 45° and 20°. In each case, which is the denser medium?
62. If the critical angle of a certain substance is 60°, what is the index of refraction?
63. If the index of refraction for diamond is 2.5, what is the critical angle?
64. If the index of refraction for two media is f , at what angle will a ray incident at 60** be refracted in passing from the rarer to the denser medium .?
65. There is a glass prism the faces of which make angles of 60° with each other. What angle must a ray of light make with one of these faces to receive the least possible refraction? [Index of refraction of the glass 1.6.]
66. The critical angle of a certain medium is 45°. A ray of light passing through this medium makes an angle of 30° with a perpendicular to the surface. What angle will it make with the perpendicular after emerging from the medium.?
67. A prism has angles which are each 60° and its index of refraction is 1.6. A ray of light enters the prism at an angle of 45° with one of its faces. What is the angle which the emergent ray forms with the opposite face?
68. Show that the image formed by a convex lens may be either larger or smaller than the object.
69. Prove that, when the distance of an object from a convex lens is twice the focal length, the image is at the same distance on the other §ide,
70. Plot the image of an object placed inside the principal focus of a double convex lens.
71. A rod 5 cm. long held in front of a convex lens forms an image 25 cm. long upon a screen 100 cm. from the lens. What is the focal length of the lens?
72. A convex lens 10 ft. from a screen throws a distinct image of a certain object on the screen. If the image is 5 times as large as the object, (\emph{a}) what is the focal length of the lens? (\emph{b}) what is the distance of the object from it?
73. A lamp is placed 12 ft. from a screen, and it is found that when a convex lens is placed 3 ft. from the lamp a sharp image of the lamp is thrown upon the screen. What is the focal length of the lens?
74. An object 4 cm. long is placed 20 cm. from a lens, and a real image is formed 10 cm. from the lens. If the object is placed 10 cm. from the lens, (\emph{a}) where will the image be? (\emph{b}) how large 1
75. The lens in a camera has a focal length of 15 in. How far from the lens must an object be in order that a clearly defined image of it may be thrown on a sensitive plate which is 16 in. from the lens?
76. At what distance from a convex lens of focal length F must an object be placed that the image may be J as large as the object.?
77. A wafer 1 cm. in diameter is held 10 cm. from a convex lens and on its principal axis. If the focal length of the lens is 50 cm., find the position and size of the image.
78. With a convex lens of 15 cm. focal length, where must the object be placed to form an image 4 times as large as the object (\emph{a}) for a real image? (\emph{b}) for a virtual?
79. If a 10-cent piece is 18 mm. in diameter and a silver dollar 38 mm. in diameter, at what distance in front of a convex lens whose focal length is 10 cm. must the 10-cent piece be placed to form an image as large as the silver dollar?
80. Where must an object be placed to form a real image lo ft. away from a convex lens the focal length of which is 1 ft.?
81. The conjugate foci of a lens are 20 and 30 cm. distant from the lens on opposite sides. How far from the lens would the rays of the sun be focused?
82. If when an object is placed 4 in. from a common magnifying glass it appears to be magnified 4 times, what is the focal length of the magnifying glass?
83. Prove by means of a diagram that an object of appreciable size placed farther than the focal length in front of a convex lens of equally curved surfaces will form an image concave toward the lens.
84. Prove that if the object in the previous problem is placed between the principal focus and the lens, the image will be convex toward the lens.
85. Deduce the facts proved in the previous problems concerning the curvature of the image from the equations $$\frac{1}{D_o}+\frac{1}{D_i}=\frac{1}{F} \ \text{and}\ \frac{1}{D_o}-\frac{1}{D_i}=\frac{1}{F}$$
86. An object is placed 10 cm. in front of a convex lens the focal length of which is 6 cm. If now the object is moved 1 cm. nearer the lens, what will be the distance traveled by the image?
87. A convex lens the focal length of which is 2 ft. is to be used to throw a picture on a screen 20 ft. from the lens. How far from the lens must the glass slide be placed?
88. Two of the conjugate foci of a convex lens are 10 and 15 cm. respectively from the lens. If owing to a change in the lens the conjugate foci become 8 and 20 cm. from the lens, how has the focal length of the lens been affected?