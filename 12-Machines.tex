MACHINES
Disregard friction and slipping of belts. «'=3^.
1. The pulley on the heads tock qti a lathe is 3 in. in diameter. This is belted to an 8-in. pulley on a shaft that makes 420 revolutions per minute. At what rate will a block of wood placed in the chuck revolve?
2. The pulley on the armature shaft of a dynamo is 4 in. in diameter. This is to be belted to a driving shaft which makes 500 revolutions per minute. The speed of the dynamo must be 1700 revolutions per minute. What must be the size of the pulley placed on the shaft?
3. If the wheel of a bicycle is 28 in. in diameter, the small sprocket 3 in., and the driving sprocket 8 in., how many times will a man need to move his right foot up and down in going a mile?
4. A shaft has upon it two pulleys, each 8 in. in diameter. The speed of the shaft is 400 revolutions per minute. What must be the size of the pulleys of two machines if when belted to the shaft one of them has a speed of 300 revolutions per minute and the other of 900?
6. A shaft which makes 300 revolutions per minute has a 6-in. pulley upon it. This pulley is belted to the pulley on a drum 3 ft. in diameter. If the drum is to wind off 1000 yds. of yarn per minute, what must be the size of its pulley.^
6. If the wheels of an electric car are 2 ft., the axle cogwheel 8 in., and the cog-wheel attached to the motor 12 in. in diameter, what must be the speed of the motor to carry the car a mile in 5 minutes?
7. Four cog-wheels, 10, 8, 6, and 4 in. in diameter, are placed in the same vertical plane with their cogs fitting to each other. There is a weight of 10 lbs. hung from the outside end of the horizontal diameter of the lo-in. wheel. How many pounds must push up from the outside end of the horizontal diameter of the 4-in. wheel to balance this.^
8. Four men are working at a capstan. They walk in a circle of 7 ft, diameter and ^ach exert3 ^ fprcc of 50 lbs. Each time they traverse the circle there are 2 ft. of rope pulled in. What is the resistance overcome?
9. The rope from a hand elevator is wound around an axle 4 in. in diameter. The wheel attached is 2 ft. in diameter. If a man exerts a force of 50 lbs. on the circumference of the wheel, how much weight can he support on the elevator?
10. A wheel with a diameter of 8 in. has firmly riveted to it two wheels of diameters 4 and 3 in., respectively. If weights of 16 and 20 lbs. are hung from these wheels, what weight must be hung from the large wheel to balance them?
11. The resistance offered by the water to the movement of the rudder of a boat is 200 lbs. The pilot wheel is 4 ft. in diameter and the axle 5 in. What force must the pilot apply to steer the boat?
12. At the top of a well there is arranged a wheel and axle. The wheel is 2 ft. in diameter. What must be the diameter of the axle for a man weighing 150 lbs. and pulling down 50 lbs. to lift himself and 100 lbs. of water out of the well?
13. The paddle wheel of a side- wheel boat is 10 ft. in diameter. If the resistance offered by the water to the movement of the wheel is 500 lbs., what force must be applied to a cog-wheel 3 ft. in diameter attached to the axle to propel the boat?
14. There is a bell 6 ft. high suspended by an axle 1 ft. below the top of the bell. Firmly attached to this axle is a wheel 6 ft. in diameter. The rope which rings the bell is attached to the circumference of the wheel. If the bell weighs 1000 lbs. and its center of gravity is 4 ft. from the top, how great a force must be applied to the rope to hold the bell in a horizontal position?
16. A boy arranges an inclined plane 20 ft. long and 5 ft high with a wheel and axle at the top. The diameter of the wheel is 2 ft and of the axle 8 in. He fixes a cart by a rope to the axle and by pulling on another rope which is wound around the wheel moves himself up the plane in the cart. He finds that he has to exert a force of 9 lbs. What is the resistance offered to the motion of the cart up the plane?
16. In Problem 15, how much do the boy and cart weigh?
17. In Problem 15, if the boy and cart weigh 90 lbs., with what force must the boy pull in order to move himself up the plane?
18. In Problem 15, how fast will the boy move horizontally if he pulls off 20 ft. of rope per minute?
19. The drive-wheel of a locomotive is 5 ft. in diameter. The connecting rod is attached to a pivot 8 in. from the axle. When this point of attachment is directly below the axle and a horizontal force of 1000 lbs. is exerted by the connecting rod, what will be the force that tends to move the engine in a horizontal direction?
20. On a foot lathe the rod connecting the pedal and the drive-wheel is attached 3 in. from the axle of the drivewheel. This wheel, which is 18 in. in diameter, is belted to the 2-in. headstock of the lathe. If the pedal is moved up and down 50 times in a minute, at what rate will the block in the lathe revolve?
21. If in Problem 20 the force applied in both the up and down stroke of the pedal is 20 lbs., what weight could be lifted by a string wound around the headstock?
22. The drive-wheel of a sewing-machine is 1 ft. in diameter. The pedal crank is attached i^ in. from the center of the drive-wheel. The machine-wheel which is belted to the drive-wheel is 3 in. in diameter. The needle goes up and down once for every revolution of this machine-wheel. If the operator moves her feet up and down at the rate of 100 strokes a minute, how many stitches will be made per minute?
23. If in Problem 22 the stitches are ^^ of an in. long, what will be the length of the seam sewed by the movement of the operator's feet back and forth through 20 yds.?
24. A cog-wheel 1 ft. in diameter has 21 cogs on its circumference. What must be the number of cogs on the circumference of another wheel so that when geared to this its rate of motion shall be three times as fast?
25. A turn-table 20 ft. in diameter is revolved by a small wheel 3 in. in diameter cogged to its circumference. If the turn-table is to revolve once in 2 minutes, what must be the speed of the small cog-wheel?
26. The diameter of the wheel of a copying press is 14 in. The plate is lowered 1 of an in. by every turn. If a force of 25 lbs. is applied to the wheel, what is the pressure exerted by the plate?
27. A carpenter wishes to make a bench-vise having a lever 7 in. long and a screw of such pitch that when he applies a force of 50 lbs. to the lever it will cause a pressure of 2200 lbs. on a block in the vise. How many threads must there be to the inch in the screw?
28. A mass of 1 ton is being raised by a jackscrew. If the lever of the jackscrew is 3 ft. long and the screw threads are 2 to the inch, what force must be applied to the handle to lift the weight?
29. If a screw has 5 threads to the inch, what must be the length of the lever arm for a force of 15 lbs. to produce a pressure of 5500 lbs.?
30. The wheel of an endless screw having 6 teeth to the inch, is 2 ft. in diameter and the axle 4 in. The crank arm of the screw is 14 in. long. What is the mechanical advantage of this machine?
31. The wheel at the top of a faucet is 2 in. in diameter and the screw has 8 threads to the inch. If a force of 5 lbs. is applied to the circumference of the wheel, what will be the pressure on the valve?
32. There is a fixed pulley at the bottom of a vertical wall and another at the top. A cord is passed around the pulleys, and a horse which pulls horizontally with a force of 1000 lbs. attached to it. How large a weight can the horse raise to the top of the wall?
33. A movable pulley is attached by a continuous cord to a fixed block containing two pulleys. What is the greatest weight that a force of 25 lbs. can lift.^
34. There are two pulley blocks, each of which contains three pulleys. What is the greatest weight that a force of 50 lbs. can be made to lift with them?
35. In Problem 34, what would be the tension on the rope?
36. What is the least number of pulleys, both fixed and movable, that are necessary to enable a force of 16 lbs. to lift a weight of 144 lbs.?
37. A man who can exert a force of 100 lbs. has two blocks of pulleys, one containing three and the other two pulleys. What is the greatest weight he can lift?
38. If a force of 100 lbs. is able by the use of two blocks of pulleys to lift a weight of 1 a ton, how many pulleys must there be in each block?
39. On a derrick there is a crank, the arm of which is 2 ft. long, which is attached to a cylinder 8 in. in diameter. The rope which winds about this cylinder is passed around two blocks of three pulleys each, one of which is fixed and the other movable. With a force of loo lbs. applied to the crank, what weight can be lifted?
40. A man weighing 150 lbs. has a rope and two simple pulleys. What is the least muscular force that he can exert to hold himself suspended from a beam by means of these appliances?
41. A spring balance suspended from the ceiling of a room has a pulley attached to it. Another spring balance is placed directly below the first and its ring fastened to the floor. A cord is attached to the hook of the second spring balance and the free end passed over the pulley. A lo-lb. weight is fastened to the free end of the cord. What is the reading on each spring balance and the tension on the string? Neglect the weights of the balance and pulley.
42. If the area of the end of the large piston of a hydraulic press is 2 sq. m. and the area of the small piston li sq. cm., what force must be applied to the small piston to produce a pressure of 100 kgm.?
43. In Problem 42, if the small piston moves through a distance of 6 cm., how far will the large piston move?
44. The pistons of a hydrostatic press are circular and have diameters of 3 in. and 3 ft., respectively. The smaller piston is worked by a lever the arms of which are 2 and 10 in. If a force of 100 lbs. is applied to the end of the lever, what weight can be supported on the larger piston?
45. A uniformly tapering wedge 12 in. long and 2 in. thick at the base is shoved with a force of 1 00 lbs. between two cakes of ice. What is the force tending to push the cakes apart.?
46. In using an air-pump it is found that the power applied to the pump handle goes through 8 times the distance the piston rises. If the area of the piston is 1 sq. cm. and the barometer stands at 76 cm., what power must be applied to work the pump when the density of the air inside the bell jar is ^ what it is outside?
47. In a suction pump the distance from the pivot to the point where the hand is applied is 20 in. and to the piston rod is 2 in. If the diameter of the piston is 4 in., what force must be applied to the handle to raise water from a well 20 ft. deep? Barometer is 30 in.
48. There is a door 2 in. thick and 2^ ft. wide. The hinges are flush with the surface of the door. A man gets his finger between the casing and the door on the hinge edge of the door. If a force of 5 lbs. is acting at right angles to the surface of the door at the free edge, with what force is his finger jammed?
49. In a wheelbarrow the distance between the axle of the wheel and the point of application of the hands to the handle-bars is 5 ft. If 75 lbs. of coal are placed in the bar-> row so that its center of gravity is i^ ft. from the axis of the wheel, and the floor of the barrow is held horizontal, what will be the pressure of the wheel on the ground? Disregard the weight of the barrow.
50. In Problem 49, if the barrow weighs 25 lbs. and its weight centers at the same point as the coal, what horizontal force must be exerted to push the barrow along? The coefficient of friction of the wheel with the ground is yV*