ENERGY
^ = 32 ft. per sec. per sec, or 980 cm. per sec. per sec.
1. A Stone weighing 50 g. is thrown with the velocity of 10 m. per second. What is its kinetic energy?
2. What is the kinetic energy of an engine weighing 100 tons moving at the rate of 40 miles per hour?
3. A body weighing 10 g. has fallen 5 seconds. What is its kinetic energy? What is its momentum?
4. A 4-0Z. bullet is shot vertically upward with a velocity of 1000 ft. per second, (\emph{a}) What is its kinetic energy when it leaves the gun? (\emph{b}) How far will it rise?
5. A mass of 1 ton is raised 1 yd. What is its potential energy?
6. A body weighing 5 lbs. has fallen 200 ft. (\emph{a}) What is its velocity? (\emph{d}) What is its kinetic energy? (\emph{c}) What is its momentum?
7. With what velocity must a lo-g. mass move in order to strike a target with a kinetic energy of 500 ergs?
8. A stone weighing 20 g. is thrown vertically upward with a velocity of 39.2 m. per second. What is its kinetic energy at the end of the second second?
9. What is the momentum and what is the kinetic energy of a 25-g. ball, free to fall, which has fallen 5 seconds?
10. Compare the momenta, velocities, and energies, respectively, of a cannon of mass " M " and its ball ot mass ** m," after discharge, if both are free to move.
11. How high must a mass of 10 lbs. be raised in order that it may have a potential energy of 500 ft. lbs.? With what velocity will it strike the ground if allowed to fall?
12. A pile driver weighs 100 lbs. and falls until it acquires a velocity of 50 ft. per second. With what energy will it strike?
13. In the previous problem with how much energy would the pile driver strike if it fell from the height of 50 ft.?
14. A bullet weighing 20 g. is shot vertically into the air and returns to its starting place in 10 seconds. With what kinetic energy did it leave the gun? Disregard resistance of the air.
16. Compare the kinetic energy of a 1-lb. mass moving at the rate of 500 ft. per second and that of a 500-lb. mass moving at the rate of 1 ft. per second.
16. A stone weighing 4 oz. falls until it acquires a velocity of 256 ft. per second, (\emph{a}) How much kinetic energy has it? (\emph{d}) How far has it fallen?
17. What is the mass of a body that has a kinetic energy of 500 ft. lbs. after falling freely for 4 seconds?
18. What must be the mass of a body that, after falling freely for 3 seconds, will strike with an energy of 360 ft. lbs.?
19. A body weighing 20 g. has a kinetic energy of looo ergs. How far would it ascend vertically?
20. A ball weighing 1 lb. has a velocity of 100 ft. per second, (\emph{a}) How far will it rise vertically? (\emph{d}) How far will it ascend if the velocity is doubled? (\emph{c}) How far will it ascend if the weight is doubled, the velocity remaining the same?
21. A mass weighing 150 lbs. starts from rest and attains a speed of 30 ft. per second. How many ft. lbs. of energy have been exerted upon it to give it this velocity?
22. What is the kinetic energy, expressed in ergs, of a mass of 1 GO g. moving at the rate of 80 cm. per second?
23. In the previous problem how many dynes of force must act upon this mass to bring it to rest in 10 seconds?
24. A body weighing 1 lb. is shot vertically into the air with a velocity of 32 ft. per second. How much kinetic energy has it at the end of J of a second?
26. A cannon ball weighing 10 lbs. is discharged with a velocity of 320 ft. per second. What is its kinetic energy in ft. lbs.?
26. A body weighing \ a lb. is thrown vertically downward with a velocity of 20 ft. per second and strikes the ground in 3 seconds. With how much energy does it strike?
27. With how much energy must a body weighing 4 oz. be thrown vertically upward in order to return to its starting place in 5 seconds?
28. A body weighing 8 g. is shot vertically upward and rises 8 seconds. What is its kinetic energy at the end of 5 seconds?
29. A body weighing 10 lbs. is shot vertically into the air and 10 seconds elapse before it returns to its starting place, (\emph{a}) How far did it travel in the entire time? (\emph{b}) How far in the 6th second? (\emph{c}) With what energy, the resistance of the air not being considered, did it strike the ground?
30. A stone weighing 2 lbs. is thrown vertically downward from the top of a tower 300 ft high and strikes the ground with an energy of 650 ft. lbs. With what velocity was it thrown?
31. A i-lb. mass is thrown vertically upward and at the end of 1 second is moving at the rate of 96 ft. per second. How much energy will it have on reaching the ground?
32. A force of lo lbs. acts for 3 seconds on a mass of 3 lbs. (\emph{a}) How much kinetic energy will the mass have? (\emph{b}) how much velocity?
33. With how much energy must a bullet weighing 20 g. be shot horizontally from a gun 4 m. above a level plane in order to strike the ground 150 m. away from the gun?
34. If a body weighing 1 lb. ascends to a certain vertical height, how far will another body weighing 2 lbs. ascend if it has the same initial energy?
36. A mass of 5 g. is acted upon by a constant force of 100 dynes for 6 seconds. What will be its energy at the end of 6 seconds?
36. If a cannon throwing a lo-lb. ball is trained to shoot vertically upward and is then discharged with enough powder to give an initial velocity of 500 ft. per second to the ball, find the kinetic energy of the ball when \ the distance to its maximum height. "^
37. In the previous problem how many horse-power of energy have been expended when the cannon ball has risen \ its maximum height 1
38. Energy is supplied to a pump at the bottom of a mine shaft 800 ft. below the surface, at the rate of 30 horse-power. How many cu. ft. of water will be raised per hour if all the energy is utilized in raising water?
39. A ball of weight 8 oz., and thrown vertically downward with a velocity of 25 ft. per second, has how much kinetic energy when it leaves the hands of the thrower? What will be its kinetic energy at a point 150 ft. from the thrower?
40. If 100 cu. ft. of water pass over a dam 10 ft. high in 1 minute, how much energy could be derived from this if all were utilized? A cu. ft. of water weighs 62.5 lbs.
41. If the diameter of the earth were doubled, its density remaining the same, with how much energy would a mass of 1 lb. falling from a height of 30 ft. above the surface strike the ground?
42. A body weighing 1 o lbs. is moving along a horizontal plane with an initial velocity of 64 ft. per second ; the coefficient of friction is J. What will its energy be 2 seconds before it stops?
43. A 5-lb. mass resting upon a frictionless horizontal plane is acted upon by a constant force which gives it an acceleration of 10 ft. per second per second. What will be its energy at the end of 8 seconds?
44. A body weighing 1 lb. started from rest and acted upon by a constant force moves 10 ft. in the 5th second. What is its energy at the end of this second?
46. A body weighing 5 lbs. and acted upon by a constant force moves 20 ft. along a frictionless horizontal plane in the sth and 6th seconds from rest, (\emph{a}) What is its acceleration? (\emph{b}) What will be its energy at the end of the Sth second from rest?
46. A body weighing 5 lbs. moves along a frictionless horizontal plane 30 ft. in the 3d and 4th seconds from rest. What is the acceleration, and what will be the energy of the body at the end of the 1 oth second?
47. A mass of 10 g. is shot along a horizontal plane with a velocity of 80 cm. per second and encounters a constant resistance of 20 dynes. What will be its energy at the end of the 4th second?
48. A body weighing 1 lb. moving along a horizontal plane with a velocity of 100 ft. per second encounters a constant resistance equal to \ its weight. What will be its energy in ft. lbs. after the resistance has acted for 3 seconds.^
49. A pendulum bob weighing 1 lb. swings to a height of 4 in. above its lowest point. With what kinetic energy does it pass this point, and what becomes of this energy as it moves on?
50. What initial velocity is necessary, in case of a projected ball, to strike a target 200 ft. above the starting place with a velocity of 1 00 ft. per second? Disregard the resistance of the air.
51. A balloon is ascending vertically at the rate of 16 ft. per second when a body weighing 1 lb. is dropped overboard. What will be the kinetic energy of the body at the end of the 1 St second?
52. A man standing on a platform 10 ft. above the ground with a 25-lb. weight in his hand jumps off. (\emph{a}) How much pressure does the weight exert upon his hand while he is falling? (\emph{b}) How much kinetic energy has it the moment he strikes the ground?
53. Two boys are passing a ball which weighs 4 oz. from end to end of a closed car 60 ft. long which is moving at the rate of 50 ft. per second. With how much greater energy must one of the boys impel the ball than the other?
54. The eaves of a house are 32 ft. from the ground. A body weighing 1 lb. starts from the ridgepole of the house, slides off and drops to the ground. The roof is 48 ft. long and rises 2 ft. in every 3 ft. of length of incline. With how much energy does the body strike the ground? The coefficient of friction is ^.
56. A body weighing 1 lb., started with an initial velocity of 16 ft. per second down a frictionless inclined plane 80 ft. long and rising 1 ft. in 5, strikes a horizontal plane having a coefficient of friction of \, What will be its energy after it has passed over 20 ft. of this plane?
56. A body weighing 2 lbs. slides 100 ft. down an inclined plane rising 3 ft. in 5. What is its energy at the bottom of the incline? Coefficient of friction is ^.
67. A body weighing 50 lbs. is projected 10 ft. up a frictionless inclined plane rising 3 ft. in 5. (\emph{a}) With what energy did the body start? (\emph{d}) With what velocity did it start?
58. With how much energy must a body weighing 1 lb. be shot up an inclined plane rising 3 ft. in 5, to move 20 ft along this plane? Coefficient of friction is J.
59. Two balls of 10 and 100 g. mass, situated 11 00 cm. distant from each other, are attracted toward each other by a constant force of 100 dynes. What will be the energy and momentum of each of these balls when they come together?
60. A body weighing 1 lb. is projected with a velocity of 100 ft. per second in the direction of a car's motion, from a car moving at the rate of 1 mile a minute. What will be its energy at the moment of projection?
61. With what energy must a body weighing 1 g. be shot vertically upward to be able to do 1000 ergs of work after rising 3 seconds?
62. A body weighing 1 lb. falls 196 ft. and strikes the ground. It rebounds with f its former energy. How many seconds will pass before it strikes the ground again?
63. With what energy must a ball weighing 10 lbs. be shot vertically upward in order to rise 200 ft. and strike a target with a velocity of 300 ft. per second? Disregard resistance of the air.
64. A body weighing 8 g. is shot vertically upward with a velocity of 490 m. per second. What is its total energy at the end of the 5th second? What is its kinetic energy at the end of the 3d second?