TENACITY AND ELASTICITY.
1. How many times as much weight will a wire which is J as long and twice as thick as another of similar material support?
2. How much must the diameter of a wire be increased to treble its strength?
3. If No. 27 spring brass wire breaks with a pull of 15 lbs., what must be the diameter of a spring brass wire that will just sustain a load of 1 00 lbs.?
4. How much load will No. 15 wire sustain if No. 27 sustains 15 lbs.?
5. If No. 25 spring brass wire breaks with a pull of 15 lbs., what is the breaking strength of No. 5? No 9? No. 16? No. 21? No. 30?
6. Using the data of the previous problem, what is the diameter of a brass wire which will just sustain (\emph{a}) a load of 100 Ibs..^ (\emph{d}) a load of 50 lbs. .5*
7. If No. 27 brass wire breaks with a pull of 15 lbs., and No. 30 iron wire with a pull of 9 lbs., what is the relative tenacity of iron and brass?
8. Using the data of the previous problem, what is the diameter of the smallest brass wire that will not be broken by two boys each pulling against the other with a force of 100 lbs..''
9. If No. 25 spring brass wire breaks with a pull of 25 lbs., and No. 30 steel wire with a pull of 18 lbs., what is the relative tenacity of brass and steel?
10. What is the relative tenacity of brass and steel if No. 26 steel wire can sustain a load of 50 lbs.? [Use data from Problem 7.]
11. What is the relative tenacity of copper and steel if No. 32 copper wire just sustains a load of 2 J lbs.? [Data from Problem 7.]
12. What loads will (\emph{a}) No. 12 brass wire sustain? (\emph{d}) No. 12 steel? [Data from Problem 7.]
13. What size of steel wire would it be necessary to use in making the links of a chain which is to sustain a weight of 100 lbs.? [Data from Problem 7.]
14. A wire the diameter of which is 1.2 mm. breaks with a pull of 10 lbs. ; another, the diameter of which is .04 mm., with a pull of 2 lbs. What is the relative tenacity of the two wires?
16. A wire .4 mm. in diameter breaks with a force of 40 lbs.; another wire .5 mm. in diameter with a force of 50 lbs. What is their relative tenacity?
16. No. 27 copper wire breaks with a pull of 15 lbs. If the tenacities of copper and iron are to each other as 3 : 2, find the breaking strength of No. 26 iron wire.
17. If the tenacities of silver and iron are to each other as 1 : 2, what will be the relative diameter of wires of these metals which sustain the same weights?
18. What must be the diameter of a brass wire in order that it may support the same weight as a platinum wire .8 mm. in diameter? The tenacities of brass and platinum are as 3 : 1 .
19. The tenacities of two wires are to each other as 3:2. A wire of the first kind .25 mm. in diameter breaks with a pull of 10 lbs. How much pull will it take to break a wire of the second kind .35 mm. in diameter?
20. If the pull on a wire be doubled and its diameter trebled, how will its stretch be affected?
21. The pull on two wires being the same, what will be their relative stretches if one of them is twice as long and J as thick as the other?
22. If a wire 1 m. long and 1 mm. in diameter be stretched 1 mm. by a force of 3 lbs., how far will it be stretched by a force of 4 kg.?
23. A force of 3 lbs. stretches 1 mm. a wire that is 1 m. long and .1 mm. in diameter. How much force will it take to stretch 5 mm. a wire of the same material 4 m. long and . 1 5 mm. in diameter?
24. Using data in the previous problem, how much will a force of 10 lbs. stretch a wire 5 m. long and .5 mm. in diameter?
25. What will be the diameter of a wire 3 m. long that will stretch 4 mm. when pulled by a force of 8 lbs., if a wire 1 m. long and .1 mm. in diameter stretches 1 mm. when pulled by a force of 3 lbs.?
26. A piece of No. 18 wire 3 m. long is stretched 1 cm. by a certain force. How long a piece of No. 24 wire will be stretched twice as much by the same force?
27. Hoyf much greater pull must be exerted upon a piece of ^fo.^2o brass wire 5 m. long than upon a piece of No. 27 wire 3 m. long, in order to stretch them the same amount?
28. What is the ratio between the force necessary to stretch 5 m. of No. 24 wire .4 cm. and the force needed to stretch an equal length of No. 32 wire the same distance?
29. The lengths of two wires are to each other as 4 : 5 and their diameters as 3:2. What will be their relative amount of stretch under the same pull?
30. A piece of No? 17 and a piece of No. 26 iron wire stretch the same amount under the same pull. What are the relative lengths?
31. If a force of 2 lbs. stretches 2 mm. a wire which is 1 m. long and . 1 sq. mm. in cross-section, how great a force is needed to stretch 5 mm. a wire of like material 10 m. long and 1 sq. mm. in cross-section?
32. A wire 4 m. long and .36 mm. in diameter is stretched 1 cm. by a pull of 5 lbs. How much will a wire 3 m. long and .25 mm. in diameter be stretched by a pull of 6 lbs.?
33. Using data in previous problem, how much pull will be necessary to stretch 4 cm. a wire 5 m. long and .4 mm. in diameter?
34. A wire 4 m. long and . 1 mm. in diameter is stretched 8 cm. by a force of 5 lbs. What must be the length of a wire .2 mm. in diameter that will stretch 2 cm. when pulled by a force of 10 lbs.?
35. A wire 4 m. long and .2 mm. in diameter is stretched 4 cm. by a force of 5 lbs. What must be the length of a wire .3 mm. in diameter that it may be stretched 2 cm. by a force of 10 lbs.?

\begin{center}
 \textsc{Rod of Rectangular Cross-Section}

\begin{table}[h!]
    \begin{center}
    \bgroup
    \def\arraystretch{1.2}%  1 is the default, change whatever you need
        \begin{tabular}{c|c|c|c|c|c}
            \hline
            \hline
            \textsc{Given} & Length & Width & Thickness & Load & Deflection \\
            \cline{2-6}
            \textsc{Problem} & 100 \textsc{cm} & 1 \textsc{cm} & 1 \textsc{cm} & 100 \textsc{g} & 8 \textsc{mm} \\
            \hline
            36. & 100 & 1 & 1 & 150 & ? \\
            37. & 100 & 1 & $\frac{4}{5}$ & 100 & ? \\
            38. & 100 & $1\frac{1}{3}$ & 1 & 100 & ? \\
            39. & 75 & 1 & 1 & 100 & ? \\
            40. & 100 & 1 & 1 & ? & 20 \\
            41. & 100 & 1 & ? & 100 & 27 \\
            42. & 100 & ? & 1 & 100 & 10 \\
            43. & ? & 1 & 1 & 100 & 27 \\
            44. & 100 & ? & 1.2 & 100 & 8 \\
            45. & 150 & 1 & ? & 200 & 6.75 \\
            46. & 40 & 4 & 1 & 2500 & ? \\
            47. & 350 & $3\frac{1}{2}$ & 2 & 450 & ? \\
            48. & 100 & ? & $\frac{1}{2}$ & 100 & 1.6 \\
            49. & 80 & $\frac{1}{2}$ & 4 & 2000 & ? \\
            50. & 100 & 1 & 1.2 & 200 & ? \\
            51. & 200 & 4 & $\frac{1}{2}$ & ? & 36 \\
            52. & 100 & $\frac{2}{5}$ & ? & 625 & 8 \\
            53. & 125 & $1\frac{2}{5}$ & ? & 150 & 9 \\
            54. & 225 & ? & $1\frac{1}{2}$ & 500 & 27 \\
            55. & 75 & 4 & ? & 800 & 16 \\
            \hline
            \hline
        \end{tabular}
    \egroup
    \end{center}

\end{table}

\end{center}
56. A wire 4 m. long and 2 mm. in diameter is stretched 5 mm. by a force of 10 lbs. What is the diameter of a wire 2 m. long if it is stretched 6 mm. by a force of 8 lbs.?
57. A wire 5 m. long and .4 mm. in diameter is stretched 1 2 mm. by a force of 5 lbs. What must be the diameter of a wire 3 m. long to cause it to stretch 5 mm; under a pull of 2 lbs.?
58. If a wooden rod \frac{}{} in. square, fastened at one end, is twisted 5° by forces applied at the other end, how much would it have been twisted (\emph{a}) by forces twice as great and similarly applied? (\emph{b}) by forces 3 times? (\emph{c}) 5 times? (\emph{d}) 7 times? (\emph{e}) \ times? (\emph{f}) f times? (\emph{g}) 10 times as great?
59. How much would the forces of the last problem twist a rod (\emph{a}) twice as long as that mentioned, but similar in other respects? (\emph{b}) a rod 3 times as long? (\emph{c}) f as long? (\emph{d}) J as long?
60. How much would a rod 1 in. square, but in other respects like that of Problem 58, be twisted by the same forces mentioned in that problem?
61. Assuming that the torsional stiffness of spruce wood is twice that of pine, and that a spruce rod J in. square, firmly fastened at one end, is twisted 10° by certain forces applied at the other end, (\emph{a}) how much will a pine rod, similar in other respects, but twice as long, be twisted by the same forces? (\emph{b}) how much by forces J^ as great?
62. If a pine rod similar to that of Problem 61, but 1 in. square, were subjected to forces twice as great as those above mentioned, how much would it be twisted?